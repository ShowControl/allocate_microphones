\documentclass[letterpaper]{article}
\usepackage[utf8]{inputenc}
\usepackage[T1]{fontenc}
\usepackage[english]{babel}
\usepackage{color}
\usepackage[top=0.5in,bottom=0.75in,left=0.5in,right=0.5in,includehead,head=12pt,headsep=0.2in,includefoot,foot=12pt,footskip=0.25in]{geometry}
\usepackage{multicol}
\usepackage{array}
\usepackage{longtable}
\usepackage{embedfile}
\usepackage{hyperref}
\embedfile [desc={parameters}]{test002.txt.parameters.tex}
\newcommand{\ParameterShowName}{test001}
\newcommand{\ParameterDesigner}{John Sauter}
\newcommand{\ParameterDateNumeric}{2019-09-15}
\newcommand{\ParameterDateText}{September 16, 2019}

% Pages styles
\makeatletter
\newcommand\ps@Standard{
  \renewcommand\@oddhead{Microphones for {\ParameterShowName}\hfill
    {\ParameterDateText}}
  \renewcommand\@evenhead{\@oddhead}
  \renewcommand\@oddfoot{\hfil Page {\thepage} \hfil}
  \renewcommand\@evenfoot{\@oddfoot}
  \renewcommand\thepage{\arabic{page}}
}
\makeatother
\embedfile [desc={input file to allocate\_microphones}]{test002.txt}
\pagestyle{Standard}
\setlength\tabcolsep{1mm}
\renewcommand\arraystretch{1.3}
\title{Microphones for {\ParameterShowName}}
\author{\ParameterDesigner}
\hypersetup{pdfauthor={\ParameterDesigner},
  pdftitle={Microphones for {\ParameterShowName}}}
\date{\ParameterDateNumeric}
\begin{document}
\maketitle
\tableofcontents
\newpage

In general, whenever an actor speaks, or sings by himself, he has a microphone.

\section {Assignments}
Microphones are assigned to actors as follows:

\embedfile [desc={input file}] {test002.txt.assignments.tex}
\begin{center}
\begin{longtable}{|l|m{7in}|}
\hline Mic. & Assignment \endhead \hline
1&\begin{itemize}
\item actor\_01 as character\_01 on pages 1 to 2.\end{itemize}
\\\hline
2&\begin{itemize}
\item actor\_02 as character\_02 on pages 2 to 3.\end{itemize}
\\\hline

\end{longtable}
\end{center}

\section {Microphone Moves, by Page}

\embedfile [desc={input file}] {test002.txt.microphone_moves.tex}
\begin{center}
\begin{longtable}{|l|m{7in}|}
\hline page & moves \endhead \hline
\input {test002.txt.microphone_moves.tex}
\end{longtable}
\end{center}

\section {Microphone Handling Instructions for Each Actor}

Each of these instructions is formatted to be sliced into a strip of paper
for the actor to carry in order to remember his microphone handling.
In addition, the complete list can be posted near the shoe tree that
holds the microphones.

\vskip 0.25in
\embedfile [desc={input file}] {test002.txt.actors.tex}
{\setlength{\parindent}{0in}
\input {test002.txt.actors.tex}
}

\section {List of Microphone Transactions for the Stage Manager}

\subsection {Initial Microphone Assignments}

These actors are the first to enter with their microphones:

\embedfile [desc={input file}] {test002.txt.stage_manager_before_show.tex}
\begin{center}
\begin{longtable}{|l|l|}
\hline Mic. & Actor \endhead \hline
1&actor\_01 as character\_01\\\hline

\end{longtable}
\end{center}

\subsection {Microphone Exchanges}

As the show progresses, some actors will have to share microphones. The
following list shows when microphones become free as actors leave the
stage, and when they go back into use on a different actor.

\embedfile [desc={input file}] {test002.txt.stage_manager_during_show.tex}
\begin{center}
\begin{longtable}{|l|l|l|l|}
\hline page & Mic. & Direction & Actor \endhead \hline
\input {test002.txt.stage_manager_during_show.tex}
\end{longtable}
\end{center}

\subsection {End of Show}

At the end of the show, retrieve the microphones from these actors

\embedfile [desc={input file}] {test002.txt.stage_manager_after_show.tex}
\begin{center}
\begin{longtable}{|l|l|}
\hline Mic. & Actor \endhead \hline
1&actor\_01 as character\_01\\\hline
2&actor\_02 as character\_02\\\hline

\end{longtable}
\end{center}

\subsection {Intermission}

In case we start a rehearsal at the intermission, here are the microphone
assignments at the end of intermission.  This list can also be useful as a 
cross-check during a production.

\embedfile [desc={input file}] {test002.txt.stage_manager_at_intermission.tex}
\begin{center}
\begin{longtable}{|l|l|}
\hline Mic. & Actor \endhead \hline
\input {test002.txt.stage_manager_at_intermission.tex}
\end{longtable}
\end{center}

\section {Sequencer Programming}

\subsection {Detailed list of cues}

Here are the microphone cues, with character names.  If you
have a sequencer you can use this information to program
the sequencer.  Cue names are the page number followed by a line
in quotation marks or a description of something that happens on
the stage.

\embedfile [desc={input file}] {test002.txt.microphone_switching_annotated.tex}
\subsection* {1}
\subsubsection* {Switch On}
\begin{enumerate}
\item Character character\_01 on microphone 1 using DCA 0 named Solo.
\end{enumerate}
\subsection* {2}
\subsubsection* {Switch Off}
\begin{enumerate}
\item Character character\_01 on microphone 1.
\end{enumerate}
Note that all microphones are now off.


\subsection {Cue summary}

Here are the cues again, in summary form, organized by DCA.

\embedfile [desc={input file}] {test002.txt.DCA_usage.tex}
\begin{center}
  \begin{longtable}{|p{0.75 in}|p{0.75 in}|p{0.75 in}|p{0.75 in}|p{0.75 in}|p{0.75 in}|p{0.75 in}|p{0.75 in}|p{0.75 in}|}    
    \hline prompt & DCA 0 & DCA 1 & DCA 2 & DCA 3 & DCA 4 & DCA 5 & DCA 6 & DCA 7\endhead \hline
    \raggedright 1&\centering Solo\\ (1)&&&&&&&\tabularnewline\hline
\raggedright 2&\centering Solo\\ (2)&&&&&&&\tabularnewline\hline
\raggedright 3&&&&&&&&\tabularnewline\hline

  \end{longtable}
\end{center}

\section {Shoe Tree Labels}

When not being worn by an actor, the microphones are stored in a shoe tree,
which is 4 shoes wide by 6 shoes tall.  
Each slot has a space at the top 4.75 inches wide
by 1.5 inches high.  We provide here a label for each slot showing which
actors use that microphone.

\embedfile [desc={input file}] {test002.txt.microphone_labels.tex}
{\Large
\parbox[t]{4.75in}{1\hfil\break actor\_01}\hfil\break\vskip 0.25in
\parbox[t]{4.75in}{2\hfil\break actor\_02}\hfil\break\vskip 0.25in

}

\section {Front of House operator}

The Front of House operator controls the house mixer, 
which contains a volume slider
and mute button for each microphone.  

\subsection {Microphone Labels}

Associated with each slider is space to write a name.  
These spaces are 1 inch wide by 0.75 inch tall.  
Here is a label for each space:

\embedfile [desc={input file}] {test002.txt.microphone_channels.tex}
{\Large
\input {test002.txt.microphone_channels.tex}
}

\subsection {Microphone Switching Instructions by Page}

These instructions are intended to be pasted into the script
for users of manual sound boards.

\embedfile [desc={input file}] {test002.txt.microphone_switching.tex}
\vskip 0.25in \vbox {\parbox[t]{0.5in}{1\hfil}\parbox[t]{2in}{On: 1. }}\vfil
\vskip 0.25in \vbox {\parbox[t]{0.5in}{2\hfil}\parbox[t]{2in}{On: 2. Off: 1. (Still on: 2.)}}\vfil
\vskip 0.25in \vbox {\parbox[t]{0.5in}{3\hfil}\parbox[t]{2in}{Off: 2. (All off.)}}\vfil
\vfill


\end{document}
