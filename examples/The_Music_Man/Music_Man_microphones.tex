\documentclass[letterpaper]{article}
\usepackage[utf8]{inputenc}
\usepackage[T1]{fontenc}
\usepackage[english]{babel}
\usepackage{color}
\usepackage[top=0.5in,bottom=0.75in,left=0.5in,right=0.5in,includehead,head=12pt,headsep=0.2in,includefoot,foot=12pt,footskip=0.25in]{geometry}
\usepackage{multicol}
\usepackage{array}
\usepackage{longtable}
\usepackage{embedfile}
\usepackage{hyperref}
\embedfile [desc={parameters}]{Music_Man.txt.parameters.tex}
\newcommand{\ParameterShowName}{The Music Man}
\newcommand{\ParameterVenue}{Manchester Community Theater}
\newcommand{\ParameterDesigner}{John Sauter}
\newcommand{\ParameterDateNumeric}{2019-07-08}
\newcommand{\ParameterDateText}{July 8, 2019}

% Pages styles
\makeatletter
\newcommand\ps@Standard{
  \renewcommand\@oddhead{Microphones for {\ParameterShowName}\hfill
    {\ParameterDateText}}
  \renewcommand\@evenhead{\@oddhead}
  \renewcommand\@oddfoot{\hfil Page {\thepage} \hfil}
  \renewcommand\@evenfoot{\@oddfoot}
  \renewcommand\thepage{\arabic{page}}
}
\makeatother
\embedfile [desc={input file to allocate\_microphones}]{Music_Man.txt}
\pagestyle{Standard}
\setlength\tabcolsep{1mm}
\renewcommand\arraystretch{1.3}
\title{Microphones for {\ParameterShowName}}
\author{\ParameterDesigner}
\hypersetup{pdfauthor={\ParameterDesigner},
  pdftitle={Microphones for {\ParameterShowName}}}
\date{\ParameterDateNumeric}
\begin{document}
\maketitle
\tableofcontents
\newpage

\section {Microphones}
In general, whenever an actor speaks, or sings by himself, he has a microphone.

\subsection {Assignments}
Microphones are assigned to actors as follows:

\embedfile [desc={input file}] {Music_Man.txt.assignments.tex}
\begin{center}
\begin{longtable}{|l|m{7in}|}
\hline Mic. & Assignment \endhead \hline
1&\begin{itemize}
\item Stuart Harmon as Harold and Stranger on pages 1 to 116a.\end{itemize}
\\\hline
2&\begin{itemize}
\item Ashley Hughes as Marian on pages 22b to 116a.\end{itemize}
\\\hline
3&\begin{itemize}
\item Roger Hurd as Jacey on pages 10d to 94b.\end{itemize}
\\\hline
4&\begin{itemize}
\item Russell Arrowsmith as Ewart on pages 15a to 94b.\end{itemize}
\\\hline
5&\begin{itemize}
\item Nicholas Hammes as Oliver on pages 31c to 94b.\end{itemize}
\\\hline
6&\begin{itemize}
\item Steve Hammes as Olin and contrary solo page 11 on pages 10d to 94b.\end{itemize}
\\\hline
7&\begin{itemize}
\item Tom Partridge as Charlie Cowell on pages 1 to 109a.\end{itemize}
\\\hline
8&\begin{itemize}
\item Ken Cox as Conductor and Man \#2 page 109 on pages 1 to 109a.\end{itemize}
\\\hline
9&\begin{itemize}
\item Joe Cinque as Salesman \#2 and Newspaper Reader \#1 on pages 1 to 10b;
\item Kevin Brannelly as 4th voice page 68 and Man \#2 page 114 and Man \#2 page 115 on pages 67b to 116a.\end{itemize}
\\\hline
10&\begin{itemize}
\item Gordon Goyette as Salesman \#3 and Man \#3 page 114 on pages 1 to 10b, 113d to 116a;
\item Pat Sawicki as Driver on pages 69e to 71b.\end{itemize}
\\\hline
11&\begin{itemize}
\item Mike McKnight as Salesman \#5 and Man page 35 on pages 1 to 38c;
\item Becky Tripp as 9th voice page 69 and Woman \#3 page 114 on pages 67b to 116a.\end{itemize}
\\\hline
12&\begin{itemize}
\item Bob Piotrowski as Salesman \#4 and Man \#2 page 35 and Man \#1 page 115 on pages 1 to 116a.\end{itemize}
\\\hline
13&\begin{itemize}
\item Rich Sparks as Salesman \#1 and Man \#1 page 108 and Man \#1 page 109 on pages 1 to 10b, 108b to 109a;
\item Janet Scagnelli as 11th voice page 69 on pages 67b to 71b.\end{itemize}
\\\hline
14&\begin{itemize}
\item Dave Ostrowski as Newspaper Reader \#2 and Newspaper Reader \#4 on pages 1 to 10b;
\item Jane Curran as 3rd voice page 68 and Woman \#1 page 114 on pages 67b to 116a.\end{itemize}
\\\hline
15&\begin{itemize}
\item Kevin Linkroum as Newspaper Reader \#3 and Newspaper Reader \#5 on pages 1 to 10b;
\item Kim Whitehead as 2nd voice page 67 on pages 67b to 71b.\end{itemize}
\\\hline
16&\begin{itemize}
\item Michelle Emmond as Mrs.~Paroo on pages 10c to 100b.\end{itemize}
\\\hline
17&\begin{itemize}
\item Vick Bennison as Mayor Shinn on pages 10c to 116a.\end{itemize}
\\\hline
18&\begin{itemize}
\item Christa Vordenberg as Amaryllis on pages 10c to 31b;
\item Sophie Linkroum as 8th voice page 68 on pages 67b to 71b;
\item Theresa Novek as Woman page 109 on pages 108b to 109a.\end{itemize}
\\\hline
19&\begin{itemize}
\item Mark Merrifield as Man page 11 and 1st voice page 67 and Man \#1 page 114 on pages 10c to 116a.\end{itemize}
\\\hline
20&\begin{itemize}
\item Matt O'Dowd as Boy page 11 on pages 10c to 13;
\item Robyn Holley as 5th voice page 68 on pages 67b to 71b;
\item Nancy Bennison as Woman page 115 on pages 113d to 116a.\end{itemize}
\\\hline
21&\begin{itemize}
\item Jenn Fichera as Alma on pages 10d to 116a.\end{itemize}
\\\hline
22&\begin{itemize}
\item Jenn Stover as Ethel on pages 10d to 84b.\end{itemize}
\\\hline
23&\begin{itemize}
\item Steve Taylor as Constable Locke and Farmer page 12 on pages 10d to 44d;
\item JoAnn Zall as 10th voice page 69 on pages 67b to 71b;
\item Megan Ostrowski as Boy page 77 on pages 74b to 79b.\end{itemize}
\\\hline
24&\begin{itemize}
\item Susan Abis as Wife page 12 on pages 10d to 13;
\item Janine Leffler as 6th voice page 68 on pages 67b to 71b;
\item Donna Fallon as Woman \#2 page 114 on pages 113d to 116a.\end{itemize}
\\\hline
25&\begin{itemize}
\item Mark Sousa as Marcellus on pages 13 to 113b.\end{itemize}
\\\hline
26&\begin{itemize}
\item Ben Wesenberg as Winthrop on pages 27 to 113b.\end{itemize}
\\\hline
27&\begin{itemize}
\item Meg Petersen as Eulalie on pages 31c to 109a.\end{itemize}
\\\hline
28&\begin{itemize}
\item Paula Troie as Maud and 7th voice page 68 on pages 31c to 116a.\end{itemize}
\\\hline
29&\begin{itemize}
\item Karen Hammes as Mrs.~Squires on pages 31c to 84b.\end{itemize}
\\\hline
30&\begin{itemize}
\item Patrick McKnight as Tommy on pages 31c to 79a.\end{itemize}
\\\hline
31&\begin{itemize}
\item Olivia Vordenberg as Zaneeta on pages 31c to 80b.\end{itemize}
\\\hline
32&\begin{itemize}
\item Lucy Stover as Gracie on pages 31c to 71b.\end{itemize}
\\\hline

\end{longtable}
\end{center}

\subsection {Microphone Moves, by Page}

\embedfile [desc={input file}] {Music_Man.txt.microphone_moves.tex}
\begin{center}
\begin{longtable}{|l|m{7in}|}
\hline page & moves \endhead \hline
67b&\begin{itemize}
\item Kim Whitehead as 2nd voice page 67 gets microphone 15 from Kevin Linkroum as Newspaper Reader \#3 and Newspaper Reader \#5 after page 10b;
\item Jane Curran as 3rd voice page 68 and Woman \#1 page 114 gets microphone 14 from Dave Ostrowski as Newspaper Reader \#2 and Newspaper Reader \#4 after page 10b;
\item Kevin Brannelly as 4th voice page 68 and Man \#2 page 114 and Man \#2 page 115 gets microphone 9 from Joe Cinque as Salesman \#2 and Newspaper Reader \#1 after page 10b;
\item Robyn Holley as 5th voice page 68 gets microphone 20 from Matt O'Dowd as Boy page 11 after page 13;
\item Janine Leffler as 6th voice page 68 gets microphone 24 from Susan Abis as Wife page 12 after page 13;
\item Sophie Linkroum as 8th voice page 68 gets microphone 18 from Christa Vordenberg as Amaryllis after page 31b;
\item Becky Tripp as 9th voice page 69 and Woman \#3 page 114 gets microphone 11 from Mike McKnight as Salesman \#5 and Man page 35 after page 38c;
\item JoAnn Zall as 10th voice page 69 gets microphone 23 from Steve Taylor as Constable Locke and Farmer page 12 after page 44d;
\item Janet Scagnelli as 11th voice page 69 gets microphone 13 from Rich Sparks as Salesman \#1 and Man \#1 page 108 and Man \#1 page 109 after page 10b.
\end{itemize}\\\hline
69e&\begin{itemize}
\item Pat Sawicki as Driver gets microphone 10 from Gordon Goyette as Salesman \#3 and Man \#3 page 114 after page 10b.
\end{itemize}\\\hline
72&Intermission: \begin{itemize}
\item Gordon Goyette as Salesman \#3 and Man \#3 page 114 gets microphone 10 from Pat Sawicki as Driver;
\item Rich Sparks as Salesman \#1 and Man \#1 page 108 and Man \#1 page 109 gets microphone 13 from Janet Scagnelli as 11th voice page 69;
\item Megan Ostrowski as Boy page 77 gets microphone 23 from JoAnn Zall as 10th voice page 69;
\item Theresa Novek as Woman page 109 gets microphone 18 from Sophie Linkroum as 8th voice page 68;
\item Donna Fallon as Woman \#2 page 114 gets microphone 24 from Janine Leffler as 6th voice page 68;
\item Nancy Bennison as Woman page 115 gets microphone 20 from Robyn Holley as 5th voice page 68.
\end{itemize}\\\hline

\end{longtable}
\end{center}

\subsection {Microphone Handling Instructions for Each Actor}

Each of these instructions is formatted to be sliced into a strip of paper
for the actor to carry in order to remember his microphone handling.
In addition, the complete list can be posted near the shoe tree that
holds the microphones.

\vskip 0.25in
\embedfile [desc={input file}] {Music_Man.txt.actors.tex}
{\setlength{\parindent}{0in}
\vbox {\setlength{\parindent}{0.25in}\noindent \textbf{Stuart Harmon as Harold and Stranger}: You will use microphone 1 throughout the show. \par1. \parbox[t]{7in}{Get microphone 1 from the stage manager before the show starts.}\par
2.  \parbox[t]{7in}{Return microphone 1 to the stage manager at the end of the show.}\par
\vskip 0.25in}\par 

\vbox {\setlength{\parindent}{0.25in}\noindent \textbf{Ashley Hughes as Marian}: You will use microphone 2 throughout the show. \par1. \parbox[t]{7in}{Get microphone 2 from the stage manager before the show starts.}\par
2.  \parbox[t]{7in}{Return microphone 2 to the stage manager at the end of the show.}\par
\vskip 0.25in}\par 

\vbox {\setlength{\parindent}{0.25in}\noindent \textbf{Roger Hurd as Jacey}: You will use microphone 3 throughout the show. \par1. \parbox[t]{7in}{Get microphone 3 from the stage manager before the show starts.}\par
2.  \parbox[t]{7in}{Return microphone 3 to the stage manager at the end of the show.}\par
\vskip 0.25in}\par 

\vbox {\setlength{\parindent}{0.25in}\noindent \textbf{Russell Arrowsmith as Ewart}: You will use microphone 4 throughout the show. \par1. \parbox[t]{7in}{Get microphone 4 from the stage manager before the show starts.}\par
2.  \parbox[t]{7in}{Return microphone 4 to the stage manager at the end of the show.}\par
\vskip 0.25in}\par 

\vbox {\setlength{\parindent}{0.25in}\noindent \textbf{Nicholas Hammes as Oliver}: You will use microphone 5 throughout the show. \par1. \parbox[t]{7in}{Get microphone 5 from the stage manager before the show starts.}\par
2.  \parbox[t]{7in}{Return microphone 5 to the stage manager at the end of the show.}\par
\vskip 0.25in}\par 

\vbox {\setlength{\parindent}{0.25in}\noindent \textbf{Steve Hammes as Olin and contrary solo page 11}: You will use microphone 6.  \par
1. \parbox[t]{7in}{Get microphone 6 from the stage manager before the show starts.}\par
2.  \parbox[t]{7in}{Give microphone 6 to Nancy Bennison as Woman page 115 after 37 Lida Rose (Reprise) (page 94b) but before 44 Finale --- Act 2 (page 113d).}\par
\vskip 0.25in}\par 

\vbox {\setlength{\parindent}{0.25in}\noindent \textbf{Michelle Emmond as Mrs.~Paroo}: You will use microphone 16.  \par
1. \parbox[t]{7in}{Get microphone 16 from the stage manager before the show starts.}\par
2.  \parbox[t]{7in}{Give microphone 16 to Pat Sawicki as Driver after 23 My White Knight (page 65a) but before 25 Finale Act 1 (page 69e).}\par
3.  \parbox[t]{7in}{Get microphone 16 from Pat Sawicki as Driver during intermission (page 72).}\par
4.  \parbox[t]{7in}{Return microphone 16 to the stage manager at the end of the show.}\par
\vskip 0.25in}\par 

\vbox {\setlength{\parindent}{0.25in}\noindent \textbf{Vick Bennison as Mayor Shinn}: You will use microphone 17 throughout the show. \par1. \parbox[t]{7in}{Get microphone 17 from the stage manager before the show starts.}\par
2.  \parbox[t]{7in}{Return microphone 17 to the stage manager at the end of the show.}\par
\vskip 0.25in}\par 

\vbox {\setlength{\parindent}{0.25in}\noindent \textbf{Meg Petersen as Eulalie}: You will use microphone 9.  \par
1.  \parbox[t]{7in}{Get microphone 9 from Joe Cinque as Salesman \#2 and Newspaper Reader \#1 after 03 Rock Island (page 10b) but before 09 Columbia, Gem of the Ocean (page 31c).}\par
2.  \parbox[t]{7in}{Return microphone 9 to the stage manager at the end of the show.}\par
\vskip 0.25in}\par 

\vbox {\setlength{\parindent}{0.25in}\noindent \textbf{Paula Troie as Maud and 7th voice page 68}: You will use microphone 20.  \par
1.  \parbox[t]{7in}{Get microphone 20 from Matt O'Dowd as Boy page 11 after 04 Iowa Stubborn (page 13) but before 09 Columbia, Gem of the Ocean (page 31c).}\par
2.  \parbox[t]{7in}{Return microphone 20 to the stage manager at the end of the show.}\par
\vskip 0.25in}\par 

\vbox {\setlength{\parindent}{0.25in}\noindent \textbf{Jenn Fichera as Alma}: You will use microphone 21.  \par
1. \parbox[t]{7in}{Get microphone 21 from the stage manager before the show starts.}\par
2.  \parbox[t]{7in}{Give microphone 21 to Janet Scagnelli as 11th voice page 69 after 17 Pick-a-Little, Talk-a-Little \& Goodnight, Ladies (page 53b) but before 24 The Wells Fargo Wagon (page 67b).}\par
3.  \parbox[t]{7in}{Get microphone 21 from Janet Scagnelli as 11th voice page 69 during intermission (page 72).}\par
4.  \parbox[t]{7in}{Return microphone 21 to the stage manager at the end of the show.}\par
\vskip 0.25in}\par 

\vbox {\setlength{\parindent}{0.25in}\noindent \textbf{Jenn Stover as Ethel}: You will use microphone 22.  \par
1. \parbox[t]{7in}{Get microphone 22 from the stage manager before the show starts.}\par
2.  \parbox[t]{7in}{Give microphone 22 to JoAnn Zall as 10th voice page 69 after 17 Pick-a-Little, Talk-a-Little \& Goodnight, Ladies (page 53b) but before 24 The Wells Fargo Wagon (page 67b).}\par
3.  \parbox[t]{7in}{Get microphone 22 from JoAnn Zall as 10th voice page 69 during intermission (page 72).}\par
4.  \parbox[t]{7in}{Give microphone 22 to Donna Fallon as Woman \#2 page 114 after 34 Pick-a-Little, Talk-a-Little (Reprise) (page 84b) but before 44 Finale --- Act 2 (page 113d).}\par
\vskip 0.25in}\par 

\vbox {\setlength{\parindent}{0.25in}\noindent \textbf{Karen Hammes as Mrs.~Squires}: You will use microphone 24.  \par
1.  \parbox[t]{7in}{Get microphone 24 from Susan Abis as Wife page 12 after 04 Iowa Stubborn (page 13) but before 09 Columbia, Gem of the Ocean (page 31c).}\par
2.  \parbox[t]{7in}{Give microphone 24 to Becky Tripp as 9th voice page 69 and Woman \#3 page 114 after 17 Pick-a-Little, Talk-a-Little \& Goodnight, Ladies (page 53b) but before 24 The Wells Fargo Wagon (page 67b).}\par
3.  \parbox[t]{7in}{Get microphone 24 from Becky Tripp as 9th voice page 69 and Woman \#3 page 114 during intermission (page 72).}\par
4.  \parbox[t]{7in}{Give microphone 24 to Becky Tripp as 9th voice page 69 and Woman \#3 page 114 after 34 Pick-a-Little, Talk-a-Little (Reprise) (page 84b) but before 44 Finale --- Act 2 (page 113d).}\par
\vskip 0.25in}\par 

\vbox {\setlength{\parindent}{0.25in}\noindent \textbf{Mark Sousa as Marcellus}: You will use microphone 15.  \par
1.  \parbox[t]{7in}{Get microphone 15 from Kevin Linkroum as Newspaper Reader \#3 and Newspaper Reader \#5 after 03 Rock Island (page 10b) but before 05 Ya Got Trouble (page 13).}\par
2.  \parbox[t]{7in}{Give microphone 15 to Sophie Linkroum as 8th voice page 68 after 16 The Sadder But Wiser Girl (page 48b) but before 24 The Wells Fargo Wagon (page 67b).}\par
3.  \parbox[t]{7in}{Get microphone 15 from Sophie Linkroum as 8th voice page 68 during intermission (page 72).}\par
4.  \parbox[t]{7in}{Return microphone 15 to the stage manager at the end of the show.}\par
\vskip 0.25in}\par 

\vbox {\setlength{\parindent}{0.25in}\noindent \textbf{Ben Wesenberg as Winthrop}: You will use microphone 14.  \par
1.  \parbox[t]{7in}{Get microphone 14 from Dave Ostrowski as Newspaper Reader \#2 and Newspaper Reader \#4 after 03 Rock Island (page 10b) but before 08 Goodnight, My Someone (page 27).}\par
2.  \parbox[t]{7in}{Return microphone 14 to the stage manager at the end of the show.}\par
\vskip 0.25in}\par 

\vbox {\setlength{\parindent}{0.25in}\noindent \textbf{Tom Partridge as Charlie Cowell}: You will use microphone 7.  \par
1. \parbox[t]{7in}{Get microphone 7 from the stage manager before the show starts.}\par
2.  \parbox[t]{7in}{Give microphone 7 to Robyn Holley as 5th voice page 68 after 03 Rock Island (page 10b) but before 24 The Wells Fargo Wagon (page 67b).}\par
3.  \parbox[t]{7in}{Get microphone 7 from Robyn Holley as 5th voice page 68 during intermission (page 72).}\par
4.  \parbox[t]{7in}{Return microphone 7 to the stage manager at the end of the show.}\par
\vskip 0.25in}\par 

\vbox {\setlength{\parindent}{0.25in}\noindent \textbf{Patrick McKnight as Tommy}: You will use microphone 13.  \par
1.  \parbox[t]{7in}{Get microphone 13 from Rich Sparks as Salesman \#1 and Man \#1 page 108 and Man \#1 page 109 after 03 Rock Island (page 10b) but before 09 Columbia, Gem of the Ocean (page 31c).}\par
2.  \parbox[t]{7in}{Give microphone 13 to Rich Sparks as Salesman \#1 and Man \#1 page 108 and Man \#1 page 109 after 33 Shipoopi Playoff (page 79a) but before 41 Ice Cream Sociable (page 108b).}\par
\vskip 0.25in}\par 

\vbox {\setlength{\parindent}{0.25in}\noindent \textbf{Olivia Vordenberg as Zaneeta}: You will use microphone 10.  \par
1.  \parbox[t]{7in}{Get microphone 10 from Gordon Goyette as Salesman \#3 and Man \#3 page 114 after 03 Rock Island (page 10b) but before 09 Columbia, Gem of the Ocean (page 31c).}\par
2.  \parbox[t]{7in}{Give microphone 10 to Gordon Goyette as Salesman \#3 and Man \#3 page 114 after 33 Shipoopi Playoff (page 80b) but before 44 Finale --- Act 2 (page 113d).}\par
\vskip 0.25in}\par 

\vbox {\setlength{\parindent}{0.25in}\noindent \textbf{Christa Vordenberg as Amaryllis}: You will use microphone 18.  \par
1. \parbox[t]{7in}{Get microphone 18 from the stage manager before the show starts.}\par
2.  \parbox[t]{7in}{Give microphone 18 to Kim Whitehead as 2nd voice page 67 after 08 Goodnight, My Someone (page 31b) but before 24 The Wells Fargo Wagon (page 67b).}\par
\vskip 0.25in}\par 

\vbox {\setlength{\parindent}{0.25in}\noindent \textbf{Steve Taylor as Constable Locke and Farmer page 12}: You will use microphone 23.  \par
1. \parbox[t]{7in}{Get microphone 23 from the stage manager before the show starts.}\par
2.  \parbox[t]{7in}{Give microphone 23 to Kevin Brannelly as 4th voice page 68 and Man \#2 page 114 and Man \#2 page 115 after 14 Ice Cream/Sincere (page 44d) but before 24 The Wells Fargo Wagon (page 67b).}\par
\vskip 0.25in}\par 

\vbox {\setlength{\parindent}{0.25in}\noindent \textbf{Ken Cox as Conductor and Man \#2 page 109}: You will use microphone 8.  \par
1. \parbox[t]{7in}{Get microphone 8 from the stage manager before the show starts.}\par
2.  \parbox[t]{7in}{Give microphone 8 to Lucy Stover as Gracie after 03 Rock Island (page 10b) but before 09 Columbia, Gem of the Ocean (page 31c).}\par
3.  \parbox[t]{7in}{Get microphone 8 from Lucy Stover as Gracie during intermission (page 72).}\par
4.  \parbox[t]{7in}{Return microphone 8 to the stage manager at the end of the show.}\par
\vskip 0.25in}\par 

\vbox {\setlength{\parindent}{0.25in}\noindent \textbf{Joe Cinque as Salesman \#2 and Newspaper Reader \#1}: You will use microphone 9.  \par
1. \parbox[t]{7in}{Get microphone 9 from the stage manager before the show starts.}\par
2.  \parbox[t]{7in}{Give microphone 9 to Meg Petersen as Eulalie after 03 Rock Island (page 10b) but before 09 Columbia, Gem of the Ocean (page 31c).}\par
\vskip 0.25in}\par 

\vbox {\setlength{\parindent}{0.25in}\noindent \textbf{Gordon Goyette as Salesman \#3 and Man \#3 page 114}: You will use microphone 10.  \par
1. \parbox[t]{7in}{Get microphone 10 from the stage manager before the show starts.}\par
2.  \parbox[t]{7in}{Give microphone 10 to Olivia Vordenberg as Zaneeta after 03 Rock Island (page 10b) but before 09 Columbia, Gem of the Ocean (page 31c).}\par
3.  \parbox[t]{7in}{Get microphone 10 from Olivia Vordenberg as Zaneeta after 33 Shipoopi Playoff (page 80b) but before 44 Finale --- Act 2 (page 113d).}\par
4.  \parbox[t]{7in}{Return microphone 10 to the stage manager at the end of the show.}\par
\vskip 0.25in}\par 

\vbox {\setlength{\parindent}{0.25in}\noindent \textbf{Mike McKnight as Salesman \#5 and Man page 35}: You will use microphone 11.  \par
1. \parbox[t]{7in}{Get microphone 11 from the stage manager before the show starts.}\par
2.  \parbox[t]{7in}{Give microphone 11 to Jane Curran as 3rd voice page 68 and Woman \#1 page 114 after 13 Seventy-Six Trombones --- Playoff (page 38c) but before 24 The Wells Fargo Wagon (page 67b).}\par
\vskip 0.25in}\par 

\vbox {\setlength{\parindent}{0.25in}\noindent \textbf{Bob Piotrowski as Salesman \#4 and Man \#2 page 35 and Man \#1 page 115}: You will use microphone 12.  \par
1. \parbox[t]{7in}{Get microphone 12 from the stage manager before the show starts.}\par
2.  \parbox[t]{7in}{Give microphone 12 to Janine Leffler as 6th voice page 68 after 13 Seventy-Six Trombones --- Playoff (page 38c) but before 24 The Wells Fargo Wagon (page 67b).}\par
3.  \parbox[t]{7in}{Get microphone 12 from Janine Leffler as 6th voice page 68 during intermission (page 72).}\par
4.  \parbox[t]{7in}{Return microphone 12 to the stage manager at the end of the show.}\par
\vskip 0.25in}\par 

\vbox {\setlength{\parindent}{0.25in}\noindent \textbf{Rich Sparks as Salesman \#1 and Man \#1 page 108 and Man \#1 page 109}: You will use microphone 13.  \par
1. \parbox[t]{7in}{Get microphone 13 from the stage manager before the show starts.}\par
2.  \parbox[t]{7in}{Give microphone 13 to Patrick McKnight as Tommy after 03 Rock Island (page 10b) but before 09 Columbia, Gem of the Ocean (page 31c).}\par
3.  \parbox[t]{7in}{Get microphone 13 from Patrick McKnight as Tommy after 33 Shipoopi Playoff (page 79a) but before 41 Ice Cream Sociable (page 108b).}\par
4.  \parbox[t]{7in}{Return microphone 13 to the stage manager at the end of the show.}\par
\vskip 0.25in}\par 

\vbox {\setlength{\parindent}{0.25in}\noindent \textbf{Dave Ostrowski as Newspaper Reader \#2 and Newspaper Reader \#4}: You will use microphone 14.  \par
1. \parbox[t]{7in}{Get microphone 14 from the stage manager before the show starts.}\par
2.  \parbox[t]{7in}{Give microphone 14 to Ben Wesenberg as Winthrop after 03 Rock Island (page 10b) but before 08 Goodnight, My Someone (page 27).}\par
\vskip 0.25in}\par 

\vbox {\setlength{\parindent}{0.25in}\noindent \textbf{Kevin Linkroum as Newspaper Reader \#3 and Newspaper Reader \#5}: You will use microphone 15.  \par
1. \parbox[t]{7in}{Get microphone 15 from the stage manager before the show starts.}\par
2.  \parbox[t]{7in}{Give microphone 15 to Mark Sousa as Marcellus after 03 Rock Island (page 10b) but before 05 Ya Got Trouble (page 13).}\par
\vskip 0.25in}\par 

\vbox {\setlength{\parindent}{0.25in}\noindent \textbf{Susan Abis as Wife page 12}: You will use microphone 24.  \par
1. \parbox[t]{7in}{Get microphone 24 from the stage manager before the show starts.}\par
2.  \parbox[t]{7in}{Give microphone 24 to Karen Hammes as Mrs.~Squires after 04 Iowa Stubborn (page 13) but before 09 Columbia, Gem of the Ocean (page 31c).}\par
\vskip 0.25in}\par 

\vbox {\setlength{\parindent}{0.25in}\noindent \textbf{Mark Merrifield as Man page 11 and 1st voice page 67 and Man \#1 page 114}: You will use microphone 19 throughout the show. \par1. \parbox[t]{7in}{Get microphone 19 from the stage manager before the show starts.}\par
2.  \parbox[t]{7in}{Return microphone 19 to the stage manager at the end of the show.}\par
\vskip 0.25in}\par 

\vbox {\setlength{\parindent}{0.25in}\noindent \textbf{Lucy Stover as Gracie}: You will use microphone 8.  \par
1.  \parbox[t]{7in}{Get microphone 8 from Ken Cox as Conductor and Man \#2 page 109 after 03 Rock Island (page 10b) but before 09 Columbia, Gem of the Ocean (page 31c).}\par
2.  \parbox[t]{7in}{Give microphone 8 to Ken Cox as Conductor and Man \#2 page 109 during intermission (page 72).}\par
\vskip 0.25in}\par 

\vbox {\setlength{\parindent}{0.25in}\noindent \textbf{Kim Whitehead as 2nd voice page 67}: You will use microphone 18.  \par
1.  \parbox[t]{7in}{Get microphone 18 from Christa Vordenberg as Amaryllis after 08 Goodnight, My Someone (page 31b) but before 24 The Wells Fargo Wagon (page 67b).}\par
2.  \parbox[t]{7in}{Give microphone 18 to Megan Ostrowski as Boy page 77 during intermission (page 72).}\par
\vskip 0.25in}\par 

\vbox {\setlength{\parindent}{0.25in}\noindent \textbf{Jane Curran as 3rd voice page 68 and Woman \#1 page 114}: You will use microphone 11.  \par
1.  \parbox[t]{7in}{Get microphone 11 from Mike McKnight as Salesman \#5 and Man page 35 after 13 Seventy-Six Trombones --- Playoff (page 38c) but before 24 The Wells Fargo Wagon (page 67b).}\par
2.  \parbox[t]{7in}{Return microphone 11 to the stage manager at the end of the show.}\par
\vskip 0.25in}\par 

\vbox {\setlength{\parindent}{0.25in}\noindent \textbf{Kevin Brannelly as 4th voice page 68 and Man \#2 page 114 and Man \#2 page 115}: You will use microphone 23.  \par
1.  \parbox[t]{7in}{Get microphone 23 from Steve Taylor as Constable Locke and Farmer page 12 after 14 Ice Cream/Sincere (page 44d) but before 24 The Wells Fargo Wagon (page 67b).}\par
2.  \parbox[t]{7in}{Return microphone 23 to the stage manager at the end of the show.}\par
\vskip 0.25in}\par 

\vbox {\setlength{\parindent}{0.25in}\noindent \textbf{Robyn Holley as 5th voice page 68}: You will use microphone 7.  \par
1.  \parbox[t]{7in}{Get microphone 7 from Tom Partridge as Charlie Cowell after 03 Rock Island (page 10b) but before 24 The Wells Fargo Wagon (page 67b).}\par
2.  \parbox[t]{7in}{Give microphone 7 to Tom Partridge as Charlie Cowell during intermission (page 72).}\par
\vskip 0.25in}\par 

\vbox {\setlength{\parindent}{0.25in}\noindent \textbf{Janine Leffler as 6th voice page 68}: You will use microphone 12.  \par
1.  \parbox[t]{7in}{Get microphone 12 from Bob Piotrowski as Salesman \#4 and Man \#2 page 35 and Man \#1 page 115 after 13 Seventy-Six Trombones --- Playoff (page 38c) but before 24 The Wells Fargo Wagon (page 67b).}\par
2.  \parbox[t]{7in}{Give microphone 12 to Bob Piotrowski as Salesman \#4 and Man \#2 page 35 and Man \#1 page 115 during intermission (page 72).}\par
\vskip 0.25in}\par 

\vbox {\setlength{\parindent}{0.25in}\noindent \textbf{Sophie Linkroum as 8th voice page 68}: You will use microphone 15.  \par
1.  \parbox[t]{7in}{Get microphone 15 from Mark Sousa as Marcellus after 16 The Sadder But Wiser Girl (page 48b) but before 24 The Wells Fargo Wagon (page 67b).}\par
2.  \parbox[t]{7in}{Give microphone 15 to Mark Sousa as Marcellus during intermission (page 72).}\par
\vskip 0.25in}\par 

\vbox {\setlength{\parindent}{0.25in}\noindent \textbf{Becky Tripp as 9th voice page 69 and Woman \#3 page 114}: You will use microphone 24.  \par
1.  \parbox[t]{7in}{Get microphone 24 from Karen Hammes as Mrs.~Squires after 17 Pick-a-Little, Talk-a-Little \& Goodnight, Ladies (page 53b) but before 24 The Wells Fargo Wagon (page 67b).}\par
2.  \parbox[t]{7in}{Give microphone 24 to Karen Hammes as Mrs.~Squires during intermission (page 72).}\par
3.  \parbox[t]{7in}{Get microphone 24 from Karen Hammes as Mrs.~Squires after 34 Pick-a-Little, Talk-a-Little (Reprise) (page 84b) but before 44 Finale --- Act 2 (page 113d).}\par
4.  \parbox[t]{7in}{Return microphone 24 to the stage manager at the end of the show.}\par
\vskip 0.25in}\par 

\vbox {\setlength{\parindent}{0.25in}\noindent \textbf{JoAnn Zall as 10th voice page 69}: You will use microphone 22.  \par
1.  \parbox[t]{7in}{Get microphone 22 from Jenn Stover as Ethel after 17 Pick-a-Little, Talk-a-Little \& Goodnight, Ladies (page 53b) but before 24 The Wells Fargo Wagon (page 67b).}\par
2.  \parbox[t]{7in}{Give microphone 22 to Jenn Stover as Ethel during intermission (page 72).}\par
\vskip 0.25in}\par 

\vbox {\setlength{\parindent}{0.25in}\noindent \textbf{Janet Scagnelli as 11th voice page 69}: You will use microphone 21.  \par
1.  \parbox[t]{7in}{Get microphone 21 from Jenn Fichera as Alma after 17 Pick-a-Little, Talk-a-Little \& Goodnight, Ladies (page 53b) but before 24 The Wells Fargo Wagon (page 67b).}\par
2.  \parbox[t]{7in}{Give microphone 21 to Jenn Fichera as Alma during intermission (page 72).}\par
\vskip 0.25in}\par 

\vbox {\setlength{\parindent}{0.25in}\noindent \textbf{Matt O'Dowd as Boy page 11}: You will use microphone 20.  \par
1. \parbox[t]{7in}{Get microphone 20 from the stage manager before the show starts.}\par
2.  \parbox[t]{7in}{Give microphone 20 to Paula Troie as Maud and 7th voice page 68 after 04 Iowa Stubborn (page 13) but before 09 Columbia, Gem of the Ocean (page 31c).}\par
\vskip 0.25in}\par 

\vbox {\setlength{\parindent}{0.25in}\noindent \textbf{Megan Ostrowski as Boy page 77}: You will use microphone 18.  \par
1.  \parbox[t]{7in}{Get microphone 18 from Kim Whitehead as 2nd voice page 67 during intermission (page 72).}\par
2.  \parbox[t]{7in}{Give microphone 18 to Theresa Novek as Woman page 109 after 33 Shipoopi Playoff (page 79b) but before 41 Ice Cream Sociable (page 108b).}\par
\vskip 0.25in}\par 

\vbox {\setlength{\parindent}{0.25in}\noindent \textbf{Theresa Novek as Woman page 109}: You will use microphone 18.  \par
1.  \parbox[t]{7in}{Get microphone 18 from Megan Ostrowski as Boy page 77 after 33 Shipoopi Playoff (page 79b) but before 41 Ice Cream Sociable (page 108b).}\par
2.  \parbox[t]{7in}{Return microphone 18 to the stage manager at the end of the show.}\par
\vskip 0.25in}\par 

\vbox {\setlength{\parindent}{0.25in}\noindent \textbf{Donna Fallon as Woman \#2 page 114}: You will use microphone 22.  \par
1.  \parbox[t]{7in}{Get microphone 22 from Jenn Stover as Ethel after 34 Pick-a-Little, Talk-a-Little (Reprise) (page 84b) but before 44 Finale --- Act 2 (page 113d).}\par
2.  \parbox[t]{7in}{Return microphone 22 to the stage manager at the end of the show.}\par
\vskip 0.25in}\par 

\vbox {\setlength{\parindent}{0.25in}\noindent \textbf{Nancy Bennison as Woman page 115}: You will use microphone 6.  \par
1.  \parbox[t]{7in}{Get microphone 6 from Steve Hammes as Olin and contrary solo page 11 after 37 Lida Rose (Reprise) (page 94b) but before 44 Finale --- Act 2 (page 113d).}\par
2.  \parbox[t]{7in}{Return microphone 6 to the stage manager at the end of the show.}\par
\vskip 0.25in}\par 

\vbox {\setlength{\parindent}{0.25in}\noindent \textbf{Pat Sawicki as Driver}: You will use microphone 16.  \par
1.  \parbox[t]{7in}{Get microphone 16 from Michelle Emmond as Mrs.~Paroo after 23 My White Knight (page 65a) but before 25 Finale Act 1 (page 69e).}\par
2.  \parbox[t]{7in}{Give microphone 16 to Michelle Emmond as Mrs.~Paroo during intermission (page 72).}\par
\vskip 0.25in}\par 


}

\subsection {List of Microphone Transactions for the Stage Manager}

\subsubsection {Initial Microphone Assignments}

These actors are the first to enter with their microphones:

\embedfile [desc={input file}] {Music_Man.txt.stage_manager_before_show.tex}
\begin{center}
\begin{longtable}{|l|l|}
\hline Mic. & Actor \endhead \hline
1&Stuart Harmon as Harold and Stranger\\\hline
2&Ashley Hughes as Marian\\\hline
3&Roger Hurd as Jacey\\\hline
4&Russell Arrowsmith as Ewart\\\hline
5&Nicholas Hammes as Oliver\\\hline
6&Steve Hammes as Olin and contrary solo page 11\\\hline
7&Tom Partridge as Charlie Cowell\\\hline
8&Ken Cox as Conductor and Man \#2 page 109\\\hline
9&Joe Cinque as Salesman \#2 and Newspaper Reader \#1\\\hline
10&Gordon Goyette as Salesman \#3 and Man \#3 page 114\\\hline
11&Mike McKnight as Salesman \#5 and Man page 35\\\hline
12&Bob Piotrowski as Salesman \#4 and Man \#2 page 35 and Man \#1 page 115\\\hline
13&Rich Sparks as Salesman \#1 and Man \#1 page 108 and Man \#1 page 109\\\hline
14&Dave Ostrowski as Newspaper Reader \#2 and Newspaper Reader \#4\\\hline
15&Kevin Linkroum as Newspaper Reader \#3 and Newspaper Reader \#5\\\hline
16&Michelle Emmond as Mrs.~Paroo\\\hline
17&Vick Bennison as Mayor Shinn\\\hline
18&Christa Vordenberg as Amaryllis\\\hline
19&Mark Merrifield as Man page 11 and 1st voice page 67 and Man \#1 page 114\\\hline
20&Matt O'Dowd as Boy page 11\\\hline
21&Jenn Fichera as Alma\\\hline
22&Jenn Stover as Ethel\\\hline
23&Steve Taylor as Constable Locke and Farmer page 12\\\hline
24&Susan Abis as Wife page 12\\\hline
25&Mark Sousa as Marcellus\\\hline
26&Ben Wesenberg as Winthrop\\\hline
27&Meg Petersen as Eulalie\\\hline
28&Paula Troie as Maud and 7th voice page 68\\\hline
29&Karen Hammes as Mrs.~Squires\\\hline
30&Patrick McKnight as Tommy\\\hline
31&Olivia Vordenberg as Zaneeta\\\hline
32&Lucy Stover as Gracie\\\hline

\end{longtable}
\end{center}

\subsubsection {Microphone Exchanges}

As the show progresses, some actors will have to share microphones. The
following list shows when microphones become free as actors leave the
stage, and when they go back into use on a different actor.

\embedfile [desc={input file}] {Music_Man.txt.stage_manager_during_show.tex}
\begin{center}
\begin{longtable}{|l|l|l|l|}
\hline page & Mic. & Direction & Actor \endhead \hline
 &7&from&Tom Partridge as Charlie Cowell\\*\cline{2-4}
 &8&from&Ken Cox as Conductor and Man \#2 page 109\\*\cline{2-4}
 &9&from&Joe Cinque as Salesman \#2 and Newspaper Reader \#1\\*\cline{2-4}
10b&10&from&Gordon Goyette as Salesman \#3 and Man \#3 page 114\\*\cline{2-4}
 &13&from&Rich Sparks as Salesman \#1 and Man \#1 page 108 and Man \#1 page 109\\*\cline{2-4}
 &14&from&Dave Ostrowski as Newspaper Reader \#2 and Newspaper Reader \#4\\*\cline{2-4}
 &15&from&Kevin Linkroum as Newspaper Reader \#3 and Newspaper Reader \#5\\\hline\hline
 &15&to&Mark Sousa as Marcellus\\*\cline{2-4}
13&20&from&Matt O'Dowd as Boy page 11\\*\cline{2-4}
 &24&from&Susan Abis as Wife page 12\\\hline\hline
27&14&to&Ben Wesenberg as Winthrop\\\hline\hline
31b&18&from&Christa Vordenberg as Amaryllis\\\hline\hline
 &8&to&Lucy Stover as Gracie\\*\cline{2-4}
 &9&to&Meg Petersen as Eulalie\\*\cline{2-4}
31c&10&to&Olivia Vordenberg as Zaneeta\\*\cline{2-4}
 &13&to&Patrick McKnight as Tommy\\*\cline{2-4}
 &20&to&Paula Troie as Maud and 7th voice page 68\\*\cline{2-4}
 &24&to&Karen Hammes as Mrs.~Squires\\\hline\hline
38c&11&from&Mike McKnight as Salesman \#5 and Man page 35\\*\cline{2-4}
 &12&from&Bob Piotrowski as Salesman \#4 and Man \#2 page 35 and Man \#1 page 115\\\hline\hline
44d&23&from&Steve Taylor as Constable Locke and Farmer page 12\\\hline\hline
48b&15&from&Mark Sousa as Marcellus\\\hline\hline
 &21&from&Jenn Fichera as Alma\\*\cline{2-4}
53b&22&from&Jenn Stover as Ethel\\*\cline{2-4}
 &24&from&Karen Hammes as Mrs.~Squires\\\hline\hline
65a&16&from&Michelle Emmond as Mrs.~Paroo\\\hline\hline
 &7&to&Robyn Holley as 5th voice page 68\\*\cline{2-4}
 &11&to&Jane Curran as 3rd voice page 68 and Woman \#1 page 114\\*\cline{2-4}
 &12&to&Janine Leffler as 6th voice page 68\\*\cline{2-4}
 &15&to&Sophie Linkroum as 8th voice page 68\\*\cline{2-4}
67b&18&to&Kim Whitehead as 2nd voice page 67\\*\cline{2-4}
 &21&to&Janet Scagnelli as 11th voice page 69\\*\cline{2-4}
 &22&to&JoAnn Zall as 10th voice page 69\\*\cline{2-4}
 &23&to&Kevin Brannelly as 4th voice page 68 and Man \#2 page 114 and Man \#2 page 115\\*\cline{2-4}
 &24&to&Becky Tripp as 9th voice page 69 and Woman \#3 page 114\\\hline\hline
69e&16&to&Pat Sawicki as Driver\\\hline\hline
 &7&from&Robyn Holley as 5th voice page 68\\*\cline{2-4}
 &7&to&Tom Partridge as Charlie Cowell\\*\cline{2-4}
 &8&from&Lucy Stover as Gracie\\*\cline{2-4}
 &8&to&Ken Cox as Conductor and Man \#2 page 109\\*\cline{2-4}
 &12&from&Janine Leffler as 6th voice page 68\\*\cline{2-4}
 &12&to&Bob Piotrowski as Salesman \#4 and Man \#2 page 35 and Man \#1 page 115\\*\cline{2-4}
 &15&from&Sophie Linkroum as 8th voice page 68\\*\cline{2-4}
 &15&to&Mark Sousa as Marcellus\\*\cline{2-4}
72&16&from&Pat Sawicki as Driver\\*\cline{2-4}
 &16&to&Michelle Emmond as Mrs.~Paroo\\*\cline{2-4}
 &18&from&Kim Whitehead as 2nd voice page 67\\*\cline{2-4}
 &18&to&Megan Ostrowski as Boy page 77\\*\cline{2-4}
 &21&from&Janet Scagnelli as 11th voice page 69\\*\cline{2-4}
 &21&to&Jenn Fichera as Alma\\*\cline{2-4}
 &22&from&JoAnn Zall as 10th voice page 69\\*\cline{2-4}
 &22&to&Jenn Stover as Ethel\\*\cline{2-4}
 &24&from&Becky Tripp as 9th voice page 69 and Woman \#3 page 114\\*\cline{2-4}
 &24&to&Karen Hammes as Mrs.~Squires\\\hline\hline
79a&13&from&Patrick McKnight as Tommy\\\hline\hline
79b&18&from&Megan Ostrowski as Boy page 77\\\hline\hline
80b&10&from&Olivia Vordenberg as Zaneeta\\\hline\hline
84b&22&from&Jenn Stover as Ethel\\*\cline{2-4}
 &24&from&Karen Hammes as Mrs.~Squires\\\hline\hline
94b&6&from&Steve Hammes as Olin and contrary solo page 11\\\hline\hline
108b&13&to&Rich Sparks as Salesman \#1 and Man \#1 page 108 and Man \#1 page 109\\*\cline{2-4}
 &18&to&Theresa Novek as Woman page 109\\\hline\hline
 &6&to&Nancy Bennison as Woman page 115\\*\cline{2-4}
113d&10&to&Gordon Goyette as Salesman \#3 and Man \#3 page 114\\*\cline{2-4}
 &22&to&Donna Fallon as Woman \#2 page 114\\*\cline{2-4}
 &24&to&Becky Tripp as 9th voice page 69 and Woman \#3 page 114\\\hline\hline

\end{longtable}
\end{center}

\subsubsection {End of Show}

At the end of the show, retrieve the microphones from these actors

\embedfile [desc={input file}] {Music_Man.txt.stage_manager_after_show.tex}
\begin{center}
\begin{longtable}{|l|l|}
\hline Mic. & Actor \endhead \hline
1&Stuart Harmon as Harold and Stranger\\\hline
2&Ashley Hughes as Marian\\\hline
3&Roger Hurd as Jacey\\\hline
4&Russell Arrowsmith as Ewart\\\hline
5&Nicholas Hammes as Oliver\\\hline
6&Steve Hammes as Olin and contrary solo page 11\\\hline
7&Tom Partridge as Charlie Cowell\\\hline
8&Ken Cox as Conductor and Man \#2 page 109\\\hline
9&Kevin Brannelly as 4th voice page 68 and Man \#2 page 114 and Man \#2 page 115\\\hline
10&Gordon Goyette as Salesman \#3 and Man \#3 page 114\\\hline
11&Becky Tripp as 9th voice page 69 and Woman \#3 page 114\\\hline
12&Bob Piotrowski as Salesman \#4 and Man \#2 page 35 and Man \#1 page 115\\\hline
13&Rich Sparks as Salesman \#1 and Man \#1 page 108 and Man \#1 page 109\\\hline
14&Jane Curran as 3rd voice page 68 and Woman \#1 page 114\\\hline
15&Kim Whitehead as 2nd voice page 67\\\hline
16&Michelle Emmond as Mrs.~Paroo\\\hline
17&Vick Bennison as Mayor Shinn\\\hline
18&Theresa Novek as Woman page 109\\\hline
19&Mark Merrifield as Man page 11 and 1st voice page 67 and Man \#1 page 114\\\hline
20&Nancy Bennison as Woman page 115\\\hline
21&Jenn Fichera as Alma\\\hline
22&Jenn Stover as Ethel\\\hline
23&Megan Ostrowski as Boy page 77\\\hline
24&Donna Fallon as Woman \#2 page 114\\\hline
25&Mark Sousa as Marcellus\\\hline
26&Ben Wesenberg as Winthrop\\\hline
27&Meg Petersen as Eulalie\\\hline
28&Paula Troie as Maud and 7th voice page 68\\\hline
29&Karen Hammes as Mrs.~Squires\\\hline
30&Patrick McKnight as Tommy\\\hline
31&Olivia Vordenberg as Zaneeta\\\hline
32&Lucy Stover as Gracie\\\hline

\end{longtable}
\end{center}

\subsubsection {Intermission}

In case we start a rehearsal at the intermission, here are the microphone
assignments at the end of intermission.  This list can also be useful as a 
cross-check during a production.

\embedfile [desc={input file}] {Music_Man.txt.stage_manager_at_intermission.tex}
\begin{center}
\begin{longtable}{|l|l|}
\hline Mic. & Actor \endhead \hline
1&Stuart Harmon as Harold and Stranger\\\hline
2&Ashley Hughes as Marian\\\hline
3&Roger Hurd as Jacey\\\hline
4&Russell Arrowsmith as Ewart\\\hline
5&Nicholas Hammes as Oliver\\\hline
6&Steve Hammes as Olin and contrary solo page 11\\\hline
7&Tom Partridge as Charlie Cowell\\\hline
8&Ken Cox as Conductor and Man \#2 page 109\\\hline
9&Kevin Brannelly as 4th voice page 68 and Man \#2 page 114 and Man \#2 page 115\\\hline
10&Gordon Goyette as Salesman \#3 and Man \#3 page 114\\\hline
11&Becky Tripp as 9th voice page 69 and Woman \#3 page 114\\\hline
12&Bob Piotrowski as Salesman \#4 and Man \#2 page 35 and Man \#1 page 115\\\hline
13&Rich Sparks as Salesman \#1 and Man \#1 page 108 and Man \#1 page 109\\\hline
14&Jane Curran as 3rd voice page 68 and Woman \#1 page 114\\\hline
16&Michelle Emmond as Mrs.~Paroo\\\hline
17&Vick Bennison as Mayor Shinn\\\hline
18&Theresa Novek as Woman page 109\\\hline
19&Mark Merrifield as Man page 11 and 1st voice page 67 and Man \#1 page 114\\\hline
20&Nancy Bennison as Woman page 115\\\hline
21&Jenn Fichera as Alma\\\hline
22&Jenn Stover as Ethel\\\hline
23&Megan Ostrowski as Boy page 77\\\hline
24&Donna Fallon as Woman \#2 page 114\\\hline
25&Mark Sousa as Marcellus\\\hline
26&Ben Wesenberg as Winthrop\\\hline
27&Meg Petersen as Eulalie\\\hline
28&Paula Troie as Maud and 7th voice page 68\\\hline
29&Karen Hammes as Mrs.~Squires\\\hline
30&Patrick McKnight as Tommy\\\hline
31&Olivia Vordenberg as Zaneeta\\\hline

\end{longtable}
\end{center}

\subsection {Sequencer Programming}

Here are the microphone cues, with character names.  If you
have a sequencer you can use this information to program
the sequencer.  Cue names are the page number followed by a line
in quotation marks or a description of something that happens on
the stage.

\embedfile [desc={input file}] {Music_Man.txt.microphone_switching_annotated.tex}
\subsection* {Top of Show}
\subsubsection* {Switch On}
\begin{enumerate}
\item Character Charlie Cowell on microphone 7 using DCA 7 named Charlie Cowell.
\item Character Conductor on microphone 8 using DCA 5 named Conductor.
\item Character Salesman \#2 and Newspaper Reader \#1 on microphone 9 in group Train passengers using DCA 2 named Train passengers.
\item Character Salesman \#3 on microphone 10 in group Train passengers using DCA 2 named Train passengers.
\item Character Salesman \#5 on microphone 11 in group Train passengers using DCA 2 named Train passengers.
\item Character Salesman \#4 on microphone 12 in group Train passengers using DCA 2 named Train passengers.
\item Character Salesman \#1 on microphone 13 in group Train passengers using DCA 2 named Train passengers.
\end{enumerate}
\subsection* {2: ``Boart!  All aboardt!''}
\subsubsection* {Switch Off}
\begin{enumerate}
\item Character Conductor on microphone 8.
\end{enumerate}
\subsubsection* {Still On}
Note that the following microphones are still on:
\begin{enumerate}
\item Character Charlie Cowell on microphone 7 using DCA 7 named Charlie Cowell.
\item Character Salesman \#2 and Newspaper Reader \#1 on microphone 9 using DCA 2 named Train passengers.
\item Character Salesman \#3 on microphone 10 using DCA 2 named Train passengers.
\item Character Salesman \#5 on microphone 11 using DCA 2 named Train passengers.
\item Character Salesman \#4 on microphone 12 using DCA 2 named Train passengers.
\item Character Salesman \#1 on microphone 13 using DCA 2 named Train passengers.
\end{enumerate}
\subsection* {6: ``Hill?''}
\subsubsection* {Switch On}
\begin{enumerate}
\item Character Newspaper Reader \#2 and Newspaper Reader \#4 on microphone 14 in group Train passengers using DCA 2 named Train passengers.
\item Character Newspaper Reader \#3 and Newspaper Reader \#5 on microphone 15 in group Train passengers using DCA 2 named Train passengers.
\end{enumerate}
\subsection* {6: ``No!''}
\subsubsection* {Switch Off}
\begin{enumerate}
\item Character Newspaper Reader \#2 and Newspaper Reader \#4 on microphone 14.
\item Character Newspaper Reader \#3 and Newspaper Reader \#5 on microphone 15.
\end{enumerate}
\subsubsection* {Still On}
Note that the following microphones are still on:
\begin{enumerate}
\item Character Charlie Cowell on microphone 7 using DCA 7 named Charlie Cowell.
\item Character Salesman \#2 and Newspaper Reader \#1 on microphone 9 using DCA 2 named Train passengers.
\item Character Salesman \#3 on microphone 10 using DCA 2 named Train passengers.
\item Character Salesman \#5 on microphone 11 using DCA 2 named Train passengers.
\item Character Salesman \#4 on microphone 12 using DCA 2 named Train passengers.
\item Character Salesman \#1 on microphone 13 using DCA 2 named Train passengers.
\end{enumerate}
\subsection* {8: ``but he doesn't know the territory!''}
\subsubsection* {Switch On}
\begin{enumerate}
\item Character Conductor on microphone 8 using DCA 5 named Conductor.
\end{enumerate}
\subsection* {8: ``Seegarettes illegal in this state.~ Board!''}
\subsubsection* {Switch Off}
\begin{enumerate}
\item Character Conductor on microphone 8.
\end{enumerate}
\subsubsection* {Still On}
Note that the following microphones are still on:
\begin{enumerate}
\item Character Charlie Cowell on microphone 7 using DCA 7 named Charlie Cowell.
\item Character Salesman \#2 and Newspaper Reader \#1 on microphone 9 using DCA 2 named Train passengers.
\item Character Salesman \#3 on microphone 10 using DCA 2 named Train passengers.
\item Character Salesman \#5 on microphone 11 using DCA 2 named Train passengers.
\item Character Salesman \#4 on microphone 12 using DCA 2 named Train passengers.
\item Character Salesman \#1 on microphone 13 using DCA 2 named Train passengers.
\end{enumerate}
\subsection* {10: ``sell them neck-bowed Hawkeyes out here.''}
\subsubsection* {Switch On}
\begin{enumerate}
\item Character Stranger on microphone 1 using DCA 0 named Stranger.
\item Character Conductor on microphone 8 using DCA 5 named Conductor.
\end{enumerate}
\subsection* {10: ``I don't believe I caught your name.''}
\subsubsection* {Switch Off}
\begin{enumerate}
\item Character Stranger on microphone 1.
\item Character Charlie Cowell on microphone 7.
\item Character Conductor on microphone 8.
\item Character Salesman \#2 and Newspaper Reader \#1 on microphone 9.
\item Character Salesman \#3 on microphone 10.
\item Character Salesman \#5 on microphone 11.
\item Character Salesman \#4 on microphone 12.
\item Character Salesman \#1 on microphone 13.
\end{enumerate}
Note that all microphones are now off.
\subsection* {10: immediately following}
\subsubsection* {Switch On}
\begin{enumerate}
\item Character Boy page 11 on microphone 20 in group Boys Act 1 using DCA 4 named Boys Act 1.
\end{enumerate}
\subsection* {11: ``you really ought to give Iowa a try,''}
\subsubsection* {Switch On}
\begin{enumerate}
\item Character contrary solo page 11 on microphone 6 using DCA 2 named contrary solo page 11.
\item Character Mayor Shinn on microphone 17 using DCA 3 named Mayor Shinn.
\item Character Man page 11 on microphone 19 in group Men Act 1 using DCA 7 named Men Act 1.
\item Character Alma on microphone 21 using DCA 5 named Alma.
\item Character Ethel on microphone 22 using DCA 6 named Ethel.
\end{enumerate}
\subsection* {11: ``She wouldn't a' come anyway.''}
\subsubsection* {Switch Off}
\begin{enumerate}
\item Character contrary solo page 11 on microphone 6.
\item Character Mayor Shinn on microphone 17.
\item Character Man page 11 on microphone 19.
\item Character Boy page 11 on microphone 20.
\item Character Alma on microphone 21.
\item Character Ethel on microphone 22.
\end{enumerate}
Note that all microphones are now off.
\subsection* {12: ``If your crop should happen to die.''}
\subsubsection* {Switch On}
\begin{enumerate}
\item Character Farmer page 12 on microphone 23 in group Men Act 1 using DCA 7 named Men Act 1.
\item Character Wife page 12 on microphone 24 in group Women Act 1 using DCA 5 named Women Act 1.
\end{enumerate}
\subsection* {12: ``Even though we may not ever mention it again.''}
\subsubsection* {Switch Off}
\begin{enumerate}
\item Character Farmer page 12 on microphone 23.
\item Character Wife page 12 on microphone 24.
\end{enumerate}
Note that all microphones are now off.
\subsection* {12: ``ought to give Iowa a try.''}
\subsubsection* {Switch On}
\begin{enumerate}
\item Character Harold on microphone 1 using DCA 0 named Harold.
\item Character Jacey on microphone 3 in group Jacey using DCA 3 named Jacey.
\end{enumerate}
\subsection* {13: ``Who is late as usual.''}
\subsubsection* {Switch On}
\begin{enumerate}
\item Character Marcellus on microphone 15 using DCA 6 named Marcellus.
\end{enumerate}
\subsubsection* {Switch Off}
\begin{enumerate}
\item Character Jacey on microphone 3.
\end{enumerate}
\subsubsection* {Still On}
Note that the following microphones are still on:
\begin{enumerate}
\item Character Harold on microphone 1 using DCA 0 named Harold.
\item Character Marcellus on microphone 15 using DCA 6 named Marcellus.
\end{enumerate}
\subsection* {13: ``Music teacher.''}
\subsubsection* {Switch On}
\begin{enumerate}
\item Character Ewart on microphone 4 in group Ewart using DCA 4 named Ewart.
\end{enumerate}
\subsubsection* {Switch Off}
\begin{enumerate}
\item Character Marcellus on microphone 15.
\end{enumerate}
\subsubsection* {Still On}
Note that the following microphones are still on:
\begin{enumerate}
\item Character Harold on microphone 1 using DCA 0 named Harold.
\item Character Ewart on microphone 4 using DCA 4 named Ewart.
\end{enumerate}
\subsection* {15: ``a pool table in your community.''}
\subsubsection* {Switch Off}
\begin{enumerate}
\item Character Ewart on microphone 4.
\end{enumerate}
\subsubsection* {Still On}
Note that character Harold on microphone 1 is still on, using DCA 0 named Harold.\subsection* {22: ``The young ones moral after school''}
\subsubsection* {Switch On}
\begin{enumerate}
\item Character Marian on microphone 2 using DCA 1 named Marian.
\end{enumerate}
\subsection* {23: ``Good!''}
\subsubsection* {Switch Off}
\begin{enumerate}
\item Character Harold on microphone 1.
\item Character Marian on microphone 2.
\end{enumerate}
Note that all microphones are now off.
\subsection* {23: immediately following}
\subsubsection* {Switch On}
\begin{enumerate}
\item Character Marian on microphone 2 using DCA 1 named Marian.
\item Character Mrs.~Paroo on microphone 16 using DCA 7 named Mrs.~Paroo.
\item Character Amaryllis on microphone 18 using DCA 5 named Amaryllis.
\end{enumerate}
\subsection* {27: ``Yes, dear.''}
\subsubsection* {Switch On}
\begin{enumerate}
\item Character Winthrop on microphone 14 using DCA 4 named Winthrop.
\end{enumerate}
\subsection* {28: ``He's crying.''}
\subsubsection* {Switch Off}
\begin{enumerate}
\item Character Winthrop on microphone 14.
\item Character Mrs.~Paroo on microphone 16.
\end{enumerate}
\subsubsection* {Still On}
Note that the following microphones are still on:
\begin{enumerate}
\item Character Marian on microphone 2 using DCA 1 named Marian.
\item Character Amaryllis on microphone 18 using DCA 5 named Amaryllis.
\end{enumerate}
\subsection* {31: ``Goodnight.''}
\subsubsection* {Switch Off}
\begin{enumerate}
\item Character Marian on microphone 2.
\item Character Amaryllis on microphone 18.
\end{enumerate}
Note that all microphones are now off.
\subsection* {31: Scene five opens}
\subsubsection* {Switch On}
\begin{enumerate}
\item Character Eulalie on microphone 9 using DCA 4 named Eulalie.
\end{enumerate}
\subsection* {31: song ends}
\subsubsection* {Switch On}
\begin{enumerate}
\item Character Mayor Shinn on microphone 17 using DCA 3 named Mayor Shinn.
\end{enumerate}
\subsubsection* {Switch Off}
\begin{enumerate}
\item Character Eulalie on microphone 9.
\end{enumerate}
\subsubsection* {Still On}
Note that character Mayor Shinn on microphone 17 is still on, using DCA 3 named Mayor Shinn.\subsection* {32: ``will take a leading part.''}
\subsubsection* {Switch On}
\begin{enumerate}
\item Character Eulalie on microphone 9 using DCA 4 named Eulalie.
\end{enumerate}
\subsection* {33: ``I'm shot, George; who shot me?''}
\subsubsection* {Switch On}
\begin{enumerate}
\item Character Gracie on microphone 8 using DCA 6 named Gracie.
\item Character Alma on microphone 21 using DCA 5 named Alma.
\item Character Constable Locke on microphone 23 using DCA 4 named Constable Locke.
\end{enumerate}
\subsubsection* {Switch Off}
\begin{enumerate}
\item Character Eulalie on microphone 9.
\end{enumerate}
\subsubsection* {Still On}
Note that the following microphones are still on:
\begin{enumerate}
\item Character Gracie on microphone 8 using DCA 6 named Gracie.
\item Character Mayor Shinn on microphone 17 using DCA 3 named Mayor Shinn.
\item Character Alma on microphone 21 using DCA 5 named Alma.
\item Character Constable Locke on microphone 23 using DCA 4 named Constable Locke.
\end{enumerate}
\subsection* {33: ``Four store and seven ---''}
\subsubsection* {Switch On}
\begin{enumerate}
\item Character Jacey on microphone 3 in group Quartet using DCA 2 named Quartet.
\item Character Ewart on microphone 4 in group Quartet using DCA 2 named Quartet.
\item Character Oliver on microphone 5 in group Quartet using DCA 2 named Quartet.
\item Character Olin on microphone 6 in group Quartet using DCA 2 named Quartet.
\end{enumerate}
\subsection* {34: ``Four score ---''}
\subsubsection* {Switch On}
\begin{enumerate}
\item Character Harold on microphone 1 using DCA 0 named Harold.
\item Character Man page 35 on microphone 11 in group Men Act 1 using DCA 7 named Men Act 1.
\item Character Man \#2 page 35 on microphone 12 in group Men Act 1 using DCA 7 named Men Act 1.
\item Character Maud on microphone 20 using DCA 2 named Maud.
\end{enumerate}
\subsubsection* {Switch Off}
\begin{enumerate}
\item Character Jacey on microphone 3.
\item Character Ewart on microphone 4.
\item Character Oliver on microphone 5.
\item Character Olin on microphone 6.
\item Character Gracie on microphone 8.
\item Character Alma on microphone 21.
\item Character Constable Locke on microphone 23.
\end{enumerate}
\subsubsection* {Still On}
Note that the following microphones are still on:
\begin{enumerate}
\item Character Harold on microphone 1 using DCA 0 named Harold.
\item Character Man page 35 on microphone 11 using DCA 7 named Men Act 1.
\item Character Man \#2 page 35 on microphone 12 using DCA 7 named Men Act 1.
\item Character Mayor Shinn on microphone 17 using DCA 3 named Mayor Shinn.
\item Character Maud on microphone 20 using DCA 2 named Maud.
\end{enumerate}
\subsection* {35: ``we get this pool table matter settled!''}
\subsubsection* {Switch Off}
\begin{enumerate}
\item Character Man page 35 on microphone 11.
\item Character Man \#2 page 35 on microphone 12.
\item Character Mayor Shinn on microphone 17.
\item Character Maud on microphone 20.
\end{enumerate}
\subsubsection* {Still On}
Note that character Harold on microphone 1 is still on, using DCA 0 named Harold.\subsection* {38: Shinn corners the school board}
\subsubsection* {Switch On}
\begin{enumerate}
\item Character Tommy on microphone 13 using DCA 5 named Tommy.
\item Character Mayor Shinn on microphone 17 using DCA 3 named Mayor Shinn.
\item Character Constable Locke on microphone 23 using DCA 4 named Constable Locke.
\end{enumerate}
\subsection* {39: ``Ten o'clock sharp.''}
\subsubsection* {Switch Off}
\begin{enumerate}
\item Character Mayor Shinn on microphone 17.
\end{enumerate}
\subsubsection* {Still On}
Note that the following microphones are still on:
\begin{enumerate}
\item Character Harold on microphone 1 using DCA 0 named Harold.
\item Character Tommy on microphone 13 using DCA 5 named Tommy.
\item Character Constable Locke on microphone 23 using DCA 4 named Constable Locke.
\end{enumerate}
\subsection* {40: ``What's your name?''}
\subsubsection* {Switch On}
\begin{enumerate}
\item Character Zaneeta on microphone 10 using DCA 6 named Zaneeta.
\end{enumerate}
\subsection* {41: ``Ye Gods.''}
\subsubsection* {Switch Off}
\begin{enumerate}
\item Character Zaneeta on microphone 10.
\item Character Tommy on microphone 13.
\end{enumerate}
\subsubsection* {Still On}
Note that the following microphones are still on:
\begin{enumerate}
\item Character Harold on microphone 1 using DCA 0 named Harold.
\item Character Constable Locke on microphone 23 using DCA 4 named Constable Locke.
\end{enumerate}
\subsection* {41: ``She's the Mayor's oldest girl.''}
\subsubsection* {Switch On}
\begin{enumerate}
\item Character Jacey on microphone 3 in group Quartet using DCA 2 named Quartet.
\item Character Ewart on microphone 4 in group Quartet using DCA 2 named Quartet.
\item Character Oliver on microphone 5 in group Quartet using DCA 2 named Quartet.
\item Character Olin on microphone 6 in group Quartet using DCA 2 named Quartet.
\item Character Eulalie on microphone 9 using DCA 4 named Eulalie.
\end{enumerate}
\subsubsection* {Switch Off}
\begin{enumerate}
\item Character Constable Locke on microphone 23.
\end{enumerate}
\subsubsection* {Still On}
Note that the following microphones are still on:
\begin{enumerate}
\item Character Harold on microphone 1 using DCA 0 named Harold.
\item Character Jacey on microphone 3 using DCA 2 named Quartet.
\item Character Ewart on microphone 4 using DCA 2 named Quartet.
\item Character Oliver on microphone 5 using DCA 2 named Quartet.
\item Character Olin on microphone 6 using DCA 2 named Quartet.
\item Character Eulalie on microphone 9 using DCA 4 named Eulalie.
\end{enumerate}
\subsection* {44: ``mystify...''}
\subsubsection* {Switch Off}
\begin{enumerate}
\item Character Harold on microphone 1.
\end{enumerate}
\subsubsection* {Still On}
Note that the following microphones are still on:
\begin{enumerate}
\item Character Jacey on microphone 3 using DCA 2 named Quartet.
\item Character Ewart on microphone 4 using DCA 2 named Quartet.
\item Character Oliver on microphone 5 using DCA 2 named Quartet.
\item Character Olin on microphone 6 using DCA 2 named Quartet.
\item Character Eulalie on microphone 9 using DCA 4 named Eulalie.
\end{enumerate}
\subsection* {44: ``Oh, we're in love.''}
\subsubsection* {Switch Off}
\begin{enumerate}
\item Character Jacey on microphone 3.
\item Character Ewart on microphone 4.
\item Character Oliver on microphone 5.
\item Character Olin on microphone 6.
\item Character Eulalie on microphone 9.
\end{enumerate}
Note that all microphones are now off.
\subsection* {44: Scene six opens}
\subsubsection* {Switch On}
\begin{enumerate}
\item Character Harold on microphone 1 using DCA 0 named Harold.
\item Character Marian on microphone 2 using DCA 1 named Marian.
\end{enumerate}
\subsection* {45: ``interesting information about you.''}
\subsubsection* {Switch On}
\begin{enumerate}
\item Character Marcellus on microphone 15 using DCA 6 named Marcellus.
\end{enumerate}
\subsubsection* {Switch Off}
\begin{enumerate}
\item Character Marian on microphone 2.
\end{enumerate}
\subsubsection* {Still On}
Note that the following microphones are still on:
\begin{enumerate}
\item Character Harold on microphone 1 using DCA 0 named Harold.
\item Character Marcellus on microphone 15 using DCA 6 named Marcellus.
\end{enumerate}
\subsection* {48: ``The sadder but wiser girl for me.''}
\subsubsection* {Switch On}
\begin{enumerate}
\item Character Eulalie on microphone 9 in group pick-a-little using DCA 3 named pick-a-little.
\item Character Maud on microphone 20 in group pick-a-little using DCA 3 named pick-a-little.
\item Character Alma on microphone 21 in group pick-a-little using DCA 3 named pick-a-little.
\item Character Ethel on microphone 22 in group pick-a-little using DCA 3 named pick-a-little.
\item Character Mrs.~Squires on microphone 24 in group pick-a-little using DCA 3 named pick-a-little.
\end{enumerate}
\subsubsection* {Switch Off}
\begin{enumerate}
\item Character Marcellus on microphone 15.
\end{enumerate}
\subsubsection* {Still On}
Note that the following microphones are still on:
\begin{enumerate}
\item Character Harold on microphone 1 using DCA 0 named Harold.
\item Character Eulalie on microphone 9 using DCA 3 named pick-a-little.
\item Character Maud on microphone 20 using DCA 3 named pick-a-little.
\item Character Alma on microphone 21 using DCA 3 named pick-a-little.
\item Character Ethel on microphone 22 using DCA 3 named pick-a-little.
\item Character Mrs.~Squires on microphone 24 using DCA 3 named pick-a-little.
\end{enumerate}
\subsection* {52: ``cheep, cheep, cheep, cheep!''}
\subsubsection* {Switch On}
\begin{enumerate}
\item Character Jacey on microphone 3 in group Quartet using DCA 2 named Quartet.
\item Character Ewart on microphone 4 in group Quartet using DCA 2 named Quartet.
\item Character Oliver on microphone 5 in group Quartet using DCA 2 named Quartet.
\item Character Olin on microphone 6 in group Quartet using DCA 2 named Quartet.
\end{enumerate}
\subsection* {53: ``pick a little, talk a little, cheep!''}
\subsubsection* {Switch Off}
\begin{enumerate}
\item Character Harold on microphone 1.
\item Character Jacey on microphone 3.
\item Character Ewart on microphone 4.
\item Character Oliver on microphone 5.
\item Character Olin on microphone 6.
\item Character Eulalie on microphone 9.
\item Character Maud on microphone 20.
\item Character Alma on microphone 21.
\item Character Ethel on microphone 22.
\item Character Mrs.~Squires on microphone 24.
\end{enumerate}
Note that all microphones are now off.
\subsection* {53: Scene seven opens}
\subsubsection* {Switch On}
\begin{enumerate}
\item Character Harold on microphone 1 using DCA 0 named Harold.
\item Character Marian on microphone 2 using DCA 1 named Marian.
\end{enumerate}
\subsection* {54: ``No!''}
\subsubsection* {Switch Off}
\begin{enumerate}
\item Character Marian on microphone 2.
\end{enumerate}
\subsubsection* {Still On}
Note that character Harold on microphone 1 is still on, using DCA 0 named Harold.\subsection* {56: Marian slaps Tommy}
\subsubsection* {Switch Off}
\begin{enumerate}
\item Character Harold on microphone 1.
\end{enumerate}
Note that all microphones are now off.
\subsection* {57: Scene eight opens}
\subsubsection* {Switch On}
\begin{enumerate}
\item Character Harold on microphone 1 using DCA 0 named Harold.
\item Character Tommy on microphone 13 using DCA 5 named Tommy.
\end{enumerate}
\subsection* {57: ``G'bye Professor.''}
\subsubsection* {Switch Off}
\begin{enumerate}
\item Character Tommy on microphone 13.
\end{enumerate}
\subsubsection* {Still On}
Note that character Harold on microphone 1 is still on, using DCA 0 named Harold.\subsection* {57: doorbell rings}
\subsubsection* {Switch On}
\begin{enumerate}
\item Character Mayor Shinn on microphone 17 using DCA 3 named Mayor Shinn.
\end{enumerate}
\subsection* {59: ``a Quaker on his day off.''}
\subsubsection* {Switch Off}
\begin{enumerate}
\item Character Harold on microphone 1.
\item Character Mayor Shinn on microphone 17.
\end{enumerate}
Note that all microphones are now off.
\subsection* {59: Scene ten opens}
\subsubsection* {Switch On}
\begin{enumerate}
\item Character Harold on microphone 1 using DCA 0 named Harold.
\item Character Winthrop on microphone 14 using DCA 4 named Winthrop.
\item Character Mrs.~Paroo on microphone 16 using DCA 7 named Mrs.~Paroo.
\end{enumerate}
\subsection* {60: ``What do you think of that?''}
\subsubsection* {Switch Off}
\begin{enumerate}
\item Character Winthrop on microphone 14.
\end{enumerate}
\subsubsection* {Still On}
Note that the following microphones are still on:
\begin{enumerate}
\item Character Harold on microphone 1 using DCA 0 named Harold.
\item Character Mrs.~Paroo on microphone 16 using DCA 7 named Mrs.~Paroo.
\end{enumerate}
\subsection* {61: ``Hodado, Miss Paroo.''}
\subsubsection* {Switch On}
\begin{enumerate}
\item Character Marian on microphone 2 using DCA 1 named Marian.
\end{enumerate}
\subsection* {62: ``Well Miss Paroo, I hardly ---''}
\subsubsection* {Switch Off}
\begin{enumerate}
\item Character Marian on microphone 2.
\end{enumerate}
\subsubsection* {Still On}
Note that the following microphones are still on:
\begin{enumerate}
\item Character Harold on microphone 1 using DCA 0 named Harold.
\item Character Mrs.~Paroo on microphone 16 using DCA 7 named Mrs.~Paroo.
\end{enumerate}
\subsection* {62: ``good day to ya, Widda Paroo.''}
\subsubsection* {Switch On}
\begin{enumerate}
\item Character Marian on microphone 2 using DCA 1 named Marian.
\end{enumerate}
\subsubsection* {Switch Off}
\begin{enumerate}
\item Character Harold on microphone 1.
\end{enumerate}
\subsubsection* {Still On}
Note that the following microphones are still on:
\begin{enumerate}
\item Character Marian on microphone 2 using DCA 1 named Marian.
\item Character Mrs.~Paroo on microphone 16 using DCA 7 named Mrs.~Paroo.
\end{enumerate}
\subsection* {63: ``I know you're there.''}
\subsubsection* {Switch On}
\begin{enumerate}
\item Character Winthrop on microphone 14 using DCA 4 named Winthrop.
\end{enumerate}
\subsection* {63: ``I've written it all down.''}
\subsubsection* {Switch Off}
\begin{enumerate}
\item Character Winthrop on microphone 14.
\end{enumerate}
\subsubsection* {Still On}
Note that the following microphones are still on:
\begin{enumerate}
\item Character Marian on microphone 2 using DCA 1 named Marian.
\item Character Mrs.~Paroo on microphone 16 using DCA 7 named Mrs.~Paroo.
\end{enumerate}
\subsection* {63: ``He does?''}
\subsubsection* {Switch Off}
\begin{enumerate}
\item Character Mrs.~Paroo on microphone 16.
\end{enumerate}
\subsubsection* {Still On}
Note that character Marian on microphone 2 is still on, using DCA 1 named Marian.\subsection* {64: ```Til I die.''}
\subsubsection* {Switch Off}
\begin{enumerate}
\item Character Marian on microphone 2.
\end{enumerate}
Note that all microphones are now off.
\subsection* {65: Scene eleven opens}
\subsubsection* {Switch On}
\begin{enumerate}
\item Character Zaneeta on microphone 10 using DCA 6 named Zaneeta.
\item Character Tommy on microphone 13 using DCA 5 named Tommy.
\end{enumerate}
\subsection* {66: ``Tommy!  It's Papa!''}
\subsubsection* {Switch On}
\begin{enumerate}
\item Character Marian on microphone 2 using DCA 1 named Marian.
\item Character Eulalie on microphone 9 using DCA 4 named Eulalie.
\item Character Mayor Shinn on microphone 17 using DCA 3 named Mayor Shinn.
\end{enumerate}
\subsubsection* {Switch Off}
\begin{enumerate}
\item Character Zaneeta on microphone 10.
\item Character Tommy on microphone 13.
\end{enumerate}
\subsubsection* {Still On}
Note that the following microphones are still on:
\begin{enumerate}
\item Character Marian on microphone 2 using DCA 1 named Marian.
\item Character Eulalie on microphone 9 using DCA 4 named Eulalie.
\item Character Mayor Shinn on microphone 17 using DCA 3 named Mayor Shinn.
\end{enumerate}
\subsection* {67: ``It's on page...''}
\subsubsection* {Switch On}
\begin{enumerate}
\item Character Gracie on microphone 8 using DCA 6 named Gracie.
\end{enumerate}
\subsection* {67: ``The band instruments!''}
\subsubsection* {Switch Off}
\begin{enumerate}
\item Character Marian on microphone 2.
\item Character Gracie on microphone 8.
\item Character Eulalie on microphone 9.
\item Character Mayor Shinn on microphone 17.
\end{enumerate}
Note that all microphones are now off.
\subsection* {67: ``I wish, I wish I knew what it could be.''}
\subsubsection* {Switch On}
\begin{enumerate}
\item Character 3rd voice page 68 on microphone 11 in group WFW soloists using DCA 5 named WFW soloists.
\item Character 2nd voice page 67 on microphone 18 in group WFW soloists using DCA 5 named WFW soloists.
\item Character 1st voice page 67 on microphone 19 in group WFW soloists using DCA 5 named WFW soloists.
\item Character 4th voice page 68 on microphone 23 in group WFW soloists using DCA 5 named WFW soloists.
\end{enumerate}
\subsection* {68: ``and a cross-cut saw.''}
\subsubsection* {Switch Off}
\begin{enumerate}
\item Character 3rd voice page 68 on microphone 11.
\item Character 2nd voice page 67 on microphone 18.
\item Character 1st voice page 67 on microphone 19.
\item Character 4th voice page 68 on microphone 23.
\end{enumerate}
Note that all microphones are now off.
\subsection* {68: ``It is a prepaid surprise or C.O.D.?''}
\subsubsection* {Switch On}
\begin{enumerate}
\item Character 5th voice page 68 on microphone 7 in group WFW soloists using DCA 5 named WFW soloists.
\item Character 6th voice page 68 on microphone 12 in group WFW soloists using DCA 5 named WFW soloists.
\item Character 8th voice page 68 on microphone 15 in group WFW soloists using DCA 5 named WFW soloists.
\item Character 7th voice page 68 on microphone 20 in group WFW soloists using DCA 5 named WFW soloists.
\end{enumerate}
\subsection* {68: ``or a double boiler,''}
\subsubsection* {Switch Off}
\begin{enumerate}
\item Character 5th voice page 68 on microphone 7.
\item Character 6th voice page 68 on microphone 12.
\item Character 7th voice page 68 on microphone 20.
\end{enumerate}
\subsubsection* {Still On}
Note that character 8th voice page 68 on microphone 15 is still on, using DCA 5 named WFW soloists.\subsection* {68: ``just for me.''}
\subsubsection* {Switch Off}
\begin{enumerate}
\item Character 8th voice page 68 on microphone 15.
\end{enumerate}
Note that all microphones are now off.
\subsection* {69: ``I wish I knew what he was comin' for.''}
\subsubsection* {Switch On}
\begin{enumerate}
\item Character Jacey on microphone 3 in group Quartet using DCA 2 named Quartet.
\item Character Ewart on microphone 4 in group Quartet using DCA 2 named Quartet.
\item Character Oliver on microphone 5 in group Quartet using DCA 2 named Quartet.
\item Character Olin on microphone 6 in group Quartet using DCA 2 named Quartet.
\item Character 11th voice page 69 on microphone 21 in group WFW soloists using DCA 5 named WFW soloists.
\item Character 10th voice page 69 on microphone 22 in group WFW soloists using DCA 5 named WFW soloists.
\item Character 9th voice page 69 on microphone 24 in group WFW soloists using DCA 5 named WFW soloists.
\end{enumerate}
\subsection* {69: ``for the courthouse square.''}
\subsubsection* {Switch On}
\begin{enumerate}
\item Character Winthrop on microphone 14 using DCA 4 named Winthrop.
\end{enumerate}
\subsubsection* {Switch Off}
\begin{enumerate}
\item Character Jacey on microphone 3.
\item Character Ewart on microphone 4.
\item Character Oliver on microphone 5.
\item Character Olin on microphone 6.
\item Character 11th voice page 69 on microphone 21.
\item Character 10th voice page 69 on microphone 22.
\item Character 9th voice page 69 on microphone 24.
\end{enumerate}
\subsubsection* {Still On}
Note that character Winthrop on microphone 14 is still on, using DCA 4 named Winthrop.\subsection* {69: ``Ray-yy!''}
\subsubsection* {Switch On}
\begin{enumerate}
\item Character Driver on microphone 16 using DCA 2 named Driver.
\end{enumerate}
\subsection* {69: ``Whoa!''}
\subsubsection* {Switch Off}
\begin{enumerate}
\item Character Driver on microphone 16.
\end{enumerate}
\subsubsection* {Still On}
Note that character Winthrop on microphone 14 is still on, using DCA 4 named Winthrop.\subsection* {70: ``It'th the band insthrumenth!''}
\subsubsection* {Switch On}
\begin{enumerate}
\item Character Harold on microphone 1 using DCA 0 named Harold.
\item Character Mayor Shinn on microphone 17 using DCA 3 named Mayor Shinn.
\end{enumerate}
\subsection* {70: ``O thithter!''}
\subsubsection* {Switch On}
\begin{enumerate}
\item Character Eulalie on microphone 9 using DCA 4 named Eulalie.
\end{enumerate}
\subsubsection* {Switch Off}
\begin{enumerate}
\item Character Winthrop on microphone 14.
\end{enumerate}
\subsubsection* {Still On}
Note that the following microphones are still on:
\begin{enumerate}
\item Character Harold on microphone 1 using DCA 0 named Harold.
\item Character Eulalie on microphone 9 using DCA 4 named Eulalie.
\item Character Mayor Shinn on microphone 17 using DCA 3 named Mayor Shinn.
\end{enumerate}
\subsection* {71: ``About that book ---''}
\subsubsection* {Switch Off}
\begin{enumerate}
\item Character Eulalie on microphone 9.
\item Character Mayor Shinn on microphone 17.
\end{enumerate}
\subsubsection* {Still On}
Note that character Harold on microphone 1 is still on, using DCA 0 named Harold.\subsection* {71: ``Tuesday nights at the High School.''}
\subsubsection* {Switch Off}
\begin{enumerate}
\item Character Harold on microphone 1.
\end{enumerate}
Note that all microphones are now off.
\subsection* {73: Act two scene one opens}
\subsubsection* {Switch On}
\begin{enumerate}
\item Character Eulalie on microphone 9 using DCA 4 named Eulalie.
\end{enumerate}
\subsection* {73: ``All right, Mr.~Dunlop.''}
\subsubsection* {Switch On}
\begin{enumerate}
\item Character Jacey on microphone 3 in group Quartet using DCA 2 named Quartet.
\item Character Ewart on microphone 4 in group Quartet using DCA 2 named Quartet.
\item Character Oliver on microphone 5 in group Quartet using DCA 2 named Quartet.
\item Character Olin on microphone 6 in group Quartet using DCA 2 named Quartet.
\end{enumerate}
\subsection* {74: ``Oh, yes, ``it's you''''}
\subsubsection* {Switch On}
\begin{enumerate}
\item Character Tommy on microphone 13 using DCA 5 named Tommy.
\item Character Marcellus on microphone 15 using DCA 6 named Marcellus.
\end{enumerate}
\subsubsection* {Switch Off}
\begin{enumerate}
\item Character Jacey on microphone 3.
\item Character Ewart on microphone 4.
\item Character Oliver on microphone 5.
\item Character Olin on microphone 6.
\end{enumerate}
\subsubsection* {Still On}
Note that the following microphones are still on:
\begin{enumerate}
\item Character Eulalie on microphone 9 using DCA 4 named Eulalie.
\item Character Tommy on microphone 13 using DCA 5 named Tommy.
\item Character Marcellus on microphone 15 using DCA 6 named Marcellus.
\end{enumerate}
\subsection* {74: ``we are entitled to five more minutes.''}
\subsubsection* {Switch Off}
\begin{enumerate}
\item Character Eulalie on microphone 9.
\end{enumerate}
\subsubsection* {Still On}
Note that the following microphones are still on:
\begin{enumerate}
\item Character Tommy on microphone 13 using DCA 5 named Tommy.
\item Character Marcellus on microphone 15 using DCA 6 named Marcellus.
\end{enumerate}
\subsection* {75: ``Shipoopi!''}
\subsubsection* {Switch Off}
\begin{enumerate}
\item Character Tommy on microphone 13.
\end{enumerate}
\subsubsection* {Still On}
Note that character Marcellus on microphone 15 is still on, using DCA 6 named Marcellus.\subsection* {77: ``but you can win her yet.''}
\subsubsection* {Switch Off}
\begin{enumerate}
\item Character Marcellus on microphone 15.
\end{enumerate}
Note that all microphones are now off.
\subsection* {77: dancing concludes}
\subsubsection* {Switch On}
\begin{enumerate}
\item Character Boy page 77 on microphone 18 in group Boys Act 2 using DCA 6 named Boys Act 2.
\end{enumerate}
\subsection* {77: ``show us some new steps!''}
\subsubsection* {Switch Off}
\begin{enumerate}
\item Character Boy page 77 on microphone 18.
\end{enumerate}
Note that all microphones are now off.
\subsection* {77: ``Shipoopi!''}
\subsubsection* {Switch On}
\begin{enumerate}
\item Character Eulalie on microphone 9 using DCA 4 named Eulalie.
\item Character Zaneeta on microphone 10 using DCA 6 named Zaneeta.
\item Character Tommy on microphone 13 using DCA 5 named Tommy.
\item Character Mayor Shinn on microphone 17 using DCA 3 named Mayor Shinn.
\end{enumerate}
\subsection* {78: ``Get out, you wild kid.''}
\subsubsection* {Switch Off}
\begin{enumerate}
\item Character Tommy on microphone 13.
\end{enumerate}
\subsubsection* {Still On}
Note that the following microphones are still on:
\begin{enumerate}
\item Character Eulalie on microphone 9 using DCA 4 named Eulalie.
\item Character Zaneeta on microphone 10 using DCA 6 named Zaneeta.
\item Character Mayor Shinn on microphone 17 using DCA 3 named Mayor Shinn.
\end{enumerate}
\subsection* {79: ``My dance ---''}
\subsubsection* {Switch On}
\begin{enumerate}
\item Character Harold on microphone 1 using DCA 0 named Harold.
\item Character Marian on microphone 2 using DCA 1 named Marian.
\end{enumerate}
\subsubsection* {Switch Off}
\begin{enumerate}
\item Character Eulalie on microphone 9.
\end{enumerate}
\subsubsection* {Still On}
Note that the following microphones are still on:
\begin{enumerate}
\item Character Harold on microphone 1 using DCA 0 named Harold.
\item Character Marian on microphone 2 using DCA 1 named Marian.
\item Character Zaneeta on microphone 10 using DCA 6 named Zaneeta.
\item Character Mayor Shinn on microphone 17 using DCA 3 named Mayor Shinn.
\end{enumerate}
\subsection* {80: ``He's slipprier'n a Mississippi sturgeon!''}
\subsubsection* {Switch On}
\begin{enumerate}
\item Character Oliver on microphone 5 in group Quartet using DCA 2 named Quartet.
\end{enumerate}
\subsection* {80: ``a button hook in the well-water.''}
\subsubsection* {Switch Off}
\begin{enumerate}
\item Character Oliver on microphone 5.
\item Character Zaneeta on microphone 10.
\item Character Mayor Shinn on microphone 17.
\end{enumerate}
\subsubsection* {Still On}
Note that the following microphones are still on:
\begin{enumerate}
\item Character Harold on microphone 1 using DCA 0 named Harold.
\item Character Marian on microphone 2 using DCA 1 named Marian.
\end{enumerate}
\subsection* {82: ``Why any night this week ---''}
\subsubsection* {Switch On}
\begin{enumerate}
\item Character Maud on microphone 20 in group pick-a-little using DCA 3 named pick-a-little.
\item Character Alma on microphone 21 in group pick-a-little using DCA 3 named pick-a-little.
\item Character Ethel on microphone 22 using DCA 6 named Ethel.
\item Character Mrs.~Squires on microphone 24 in group pick-a-little using DCA 3 named pick-a-little.
\end{enumerate}
\subsubsection* {Switch Off}
\begin{enumerate}
\item Character Harold on microphone 1.
\end{enumerate}
\subsubsection* {Still On}
Note that the following microphones are still on:
\begin{enumerate}
\item Character Marian on microphone 2 using DCA 1 named Marian.
\item Character Maud on microphone 20 using DCA 3 named pick-a-little.
\item Character Alma on microphone 21 using DCA 3 named pick-a-little.
\item Character Ethel on microphone 22 using DCA 6 named Ethel.
\item Character Mrs.~Squires on microphone 24 using DCA 3 named pick-a-little.
\end{enumerate}
\subsection* {82: ``please join our Del Satre Committee''}
\subsubsection* {Change Group}
\begin{enumerate}
\item Character Alma on microphone 21 moves from group pick-a-little to group Alma using DCA 5 named Alma.
\item Character Ethel on microphone 22 moves to group pick-a-little using DCA 3 named pick-a-little.
\end{enumerate}
\subsection* {83: ``where a woman's heart should be!''}
\subsubsection* {Change Group}
\begin{enumerate}
\item Character Alma on microphone 21 moves from group Alma to group pick-a-little using DCA 3 named pick-a-little.
\end{enumerate}
\subsection* {83: ``and we simply adored them all!''}
\subsubsection* {Switch On}
\begin{enumerate}
\item Character Eulalie on microphone 9 using DCA 4 named Eulalie.
\end{enumerate}
\subsubsection* {Change Group}
\begin{enumerate}
\item Character Alma on microphone 21 moves from group pick-a-little to group Alma using DCA 5 named Alma.
\item Character Ethel on microphone 22 moves from group pick-a-little to group Ethel using DCA 6 named Ethel.
\end{enumerate}
\subsection* {83: ``Bal-zac''}
\subsubsection* {Change Group}
\begin{enumerate}
\item Character Eulalie on microphone 9 moves to group pick-a-little using DCA 3 named pick-a-little.
\item Character Alma on microphone 21 moves from group Alma to group pick-a-little using DCA 3 named pick-a-little.
\item Character Ethel on microphone 22 moves from group Ethel to group pick-a-little using DCA 3 named pick-a-little.
\end{enumerate}
\subsection* {84: (whispered) ``Cheep!''}
\subsubsection* {Switch Off}
\begin{enumerate}
\item Character Marian on microphone 2.
\item Character Eulalie on microphone 9.
\item Character Maud on microphone 20.
\item Character Alma on microphone 21.
\item Character Ethel on microphone 22.
\item Character Mrs.~Squires on microphone 24.
\end{enumerate}
Note that all microphones are now off.
\subsection* {84: Scene two opens}
\subsubsection* {Switch On}
\begin{enumerate}
\item Character Harold on microphone 1 using DCA 0 named Harold.
\item Character Jacey on microphone 3 in group Quartet using DCA 2 named Quartet.
\item Character Ewart on microphone 4 in group Quartet using DCA 2 named Quartet.
\item Character Oliver on microphone 5 in group Quartet using DCA 2 named Quartet.
\item Character Olin on microphone 6 in group Quartet using DCA 2 named Quartet.
\end{enumerate}
\subsection* {85: ``I'm home again, rose.''}
\subsubsection* {Switch Off}
\begin{enumerate}
\item Character Harold on microphone 1.
\end{enumerate}
\subsubsection* {Still On}
Note that the following microphones are still on:
\begin{enumerate}
\item Character Jacey on microphone 3 using DCA 2 named Quartet.
\item Character Ewart on microphone 4 using DCA 2 named Quartet.
\item Character Oliver on microphone 5 using DCA 2 named Quartet.
\item Character Olin on microphone 6 using DCA 2 named Quartet.
\end{enumerate}
\subsection* {85: ``Oh, Lida Rose.''}
\subsubsection* {Switch On}
\begin{enumerate}
\item Character Marian on microphone 2 using DCA 1 named Marian.
\end{enumerate}
\subsection* {87: ``Oh, Lida Rose.''}
\subsubsection* {Switch Off}
\begin{enumerate}
\item Character Jacey on microphone 3.
\item Character Ewart on microphone 4.
\item Character Oliver on microphone 5.
\item Character Olin on microphone 6.
\end{enumerate}
\subsubsection* {Still On}
Note that character Marian on microphone 2 is still on, using DCA 1 named Marian.\subsection* {87: Scene three opens}
\subsubsection* {Switch On}
\begin{enumerate}
\item Character Mrs.~Paroo on microphone 16 using DCA 7 named Mrs.~Paroo.
\end{enumerate}
\subsection* {88: ``there's nothing wrong with a ladylike hint.''}
\subsubsection* {Switch On}
\begin{enumerate}
\item Character Winthrop on microphone 14 using DCA 4 named Winthrop.
\end{enumerate}
\subsection* {90: ``La de da, La de da.''}
\subsubsection* {Switch Off}
\begin{enumerate}
\item Character Winthrop on microphone 14.
\end{enumerate}
\subsubsection* {Still On}
Note that the following microphones are still on:
\begin{enumerate}
\item Character Marian on microphone 2 using DCA 1 named Marian.
\item Character Mrs.~Paroo on microphone 16 using DCA 7 named Mrs.~Paroo.
\end{enumerate}
\subsection* {90: ``There's time later.''}
\subsubsection* {Switch On}
\begin{enumerate}
\item Character Charlie Cowell on microphone 7 using DCA 7 named Charlie Cowell.
\end{enumerate}
\subsubsection* {Switch Off}
\begin{enumerate}
\item Character Mrs.~Paroo on microphone 16.
\end{enumerate}
\subsubsection* {Still On}
Note that the following microphones are still on:
\begin{enumerate}
\item Character Marian on microphone 2 using DCA 1 named Marian.
\item Character Charlie Cowell on microphone 7 using DCA 7 named Charlie Cowell.
\end{enumerate}
\subsection* {93: ``Try me.''}
\subsubsection* {Switch On}
\begin{enumerate}
\item Character Jacey on microphone 3 in group Quartet using DCA 2 named Quartet.
\item Character Ewart on microphone 4 in group Quartet using DCA 2 named Quartet.
\item Character Oliver on microphone 5 in group Quartet using DCA 2 named Quartet.
\item Character Olin on microphone 6 in group Quartet using DCA 2 named Quartet.
\end{enumerate}
\subsection* {94: ``Neither one of you's heard the last of me, girly-girl!''}
\subsubsection* {Switch On}
\begin{enumerate}
\item Character Harold on microphone 1 using DCA 0 named Harold.
\item Character Mrs.~Paroo on microphone 16 using DCA 7 named Mrs.~Paroo.
\end{enumerate}
\subsubsection* {Switch Off}
\begin{enumerate}
\item Character Charlie Cowell on microphone 7.
\end{enumerate}
\subsubsection* {Still On}
Note that the following microphones are still on:
\begin{enumerate}
\item Character Harold on microphone 1 using DCA 0 named Harold.
\item Character Marian on microphone 2 using DCA 1 named Marian.
\item Character Jacey on microphone 3 using DCA 2 named Quartet.
\item Character Ewart on microphone 4 using DCA 2 named Quartet.
\item Character Oliver on microphone 5 using DCA 2 named Quartet.
\item Character Olin on microphone 6 using DCA 2 named Quartet.
\item Character Mrs.~Paroo on microphone 16 using DCA 7 named Mrs.~Paroo.
\end{enumerate}
\subsection* {94: ``Oh, Lida Rose.''}
\subsubsection* {Switch Off}
\begin{enumerate}
\item Character Jacey on microphone 3.
\item Character Ewart on microphone 4.
\item Character Oliver on microphone 5.
\item Character Olin on microphone 6.
\end{enumerate}
\subsubsection* {Still On}
Note that the following microphones are still on:
\begin{enumerate}
\item Character Harold on microphone 1 using DCA 0 named Harold.
\item Character Marian on microphone 2 using DCA 1 named Marian.
\item Character Mrs.~Paroo on microphone 16 using DCA 7 named Mrs.~Paroo.
\end{enumerate}
\subsection* {95: ``Well, I'll put some on.''}
\subsubsection* {Switch Off}
\begin{enumerate}
\item Character Mrs.~Paroo on microphone 16.
\end{enumerate}
\subsubsection* {Still On}
Note that the following microphones are still on:
\begin{enumerate}
\item Character Harold on microphone 1 using DCA 0 named Harold.
\item Character Marian on microphone 2 using DCA 1 named Marian.
\end{enumerate}
\subsection* {100: ``Fifteen minutes.''}
\subsubsection* {Switch On}
\begin{enumerate}
\item Character Mrs.~Paroo on microphone 16 using DCA 7 named Mrs.~Paroo.
\end{enumerate}
\subsubsection* {Switch Off}
\begin{enumerate}
\item Character Harold on microphone 1.
\end{enumerate}
\subsubsection* {Still On}
Note that the following microphones are still on:
\begin{enumerate}
\item Character Marian on microphone 2 using DCA 1 named Marian.
\item Character Mrs.~Paroo on microphone 16 using DCA 7 named Mrs.~Paroo.
\end{enumerate}
\subsection* {100: ``I been using the Think System on you from the Parlor!''}
\subsubsection* {Switch Off}
\begin{enumerate}
\item Character Marian on microphone 2.
\item Character Mrs.~Paroo on microphone 16.
\end{enumerate}
Note that all microphones are now off.
\subsection* {101: Scene four opens}
\subsubsection* {Switch On}
\begin{enumerate}
\item Character Harold on microphone 1 using DCA 0 named Harold.
\item Character Marcellus on microphone 15 using DCA 6 named Marcellus.
\end{enumerate}
\subsection* {102: ``I'll meet you at the Hotel in plenty of time.''}
\subsubsection* {Switch On}
\begin{enumerate}
\item Character Marian on microphone 2 using DCA 1 named Marian.
\end{enumerate}
\subsubsection* {Switch Off}
\begin{enumerate}
\item Character Marcellus on microphone 15.
\end{enumerate}
\subsubsection* {Still On}
Note that the following microphones are still on:
\begin{enumerate}
\item Character Harold on microphone 1 using DCA 0 named Harold.
\item Character Marian on microphone 2 using DCA 1 named Marian.
\end{enumerate}
\subsection* {104: ``there's a lot of things you don't know about me ---''}
\subsubsection* {Switch On}
\begin{enumerate}
\item Character Marcellus on microphone 15 using DCA 6 named Marcellus.
\end{enumerate}
\subsection* {104: ``Go get the rig.''}
\subsubsection* {Switch Off}
\begin{enumerate}
\item Character Marcellus on microphone 15.
\end{enumerate}
\subsubsection* {Still On}
Note that the following microphones are still on:
\begin{enumerate}
\item Character Harold on microphone 1 using DCA 0 named Harold.
\item Character Marian on microphone 2 using DCA 1 named Marian.
\end{enumerate}
\subsection* {107: ``My someone, goodnight.''}
\subsubsection* {Switch On}
\begin{enumerate}
\item Character Charlie Cowell on microphone 7 using DCA 7 named Charlie Cowell.
\item Character Marcellus on microphone 15 using DCA 6 named Marcellus.
\end{enumerate}
\subsubsection* {Switch Off}
\begin{enumerate}
\item Character Marian on microphone 2.
\end{enumerate}
\subsubsection* {Still On}
Note that the following microphones are still on:
\begin{enumerate}
\item Character Harold on microphone 1 using DCA 0 named Harold.
\item Character Charlie Cowell on microphone 7 using DCA 7 named Charlie Cowell.
\item Character Marcellus on microphone 15 using DCA 6 named Marcellus.
\end{enumerate}
\subsection* {107: ``you never even knew the territory.''}
\subsubsection* {Switch Off}
\begin{enumerate}
\item Character Charlie Cowell on microphone 7.
\end{enumerate}
\subsubsection* {Still On}
Note that the following microphones are still on:
\begin{enumerate}
\item Character Harold on microphone 1 using DCA 0 named Harold.
\item Character Marcellus on microphone 15 using DCA 6 named Marcellus.
\end{enumerate}
\subsection* {107: ``Come on! Hurry up!''}
\subsubsection* {Switch Off}
\begin{enumerate}
\item Character Harold on microphone 1.
\item Character Marcellus on microphone 15.
\end{enumerate}
Note that all microphones are now off.
\subsection* {108: Scene six opens}
\subsubsection* {Switch On}
\begin{enumerate}
\item Character Charlie Cowell on microphone 7 using DCA 7 named Charlie Cowell.
\item Character Man \#2 page 109 on microphone 8 in group Men Act 2 using DCA 6 named Men Act 2.
\item Character Eulalie on microphone 9 using DCA 4 named Eulalie.
\item Character Man \#1 page 108 and Man \#1 page 109 on microphone 13 in group Men Act 2 using DCA 6 named Men Act 2.
\item Character Mayor Shinn on microphone 17 using DCA 3 named Mayor Shinn.
\item Character Woman page 109 on microphone 18 in group Women Act 2 using DCA 5 named Women Act 2.
\end{enumerate}
\subsection* {109: ``After him!''}
\subsubsection* {Switch Off}
\begin{enumerate}
\item Character Charlie Cowell on microphone 7.
\item Character Man \#2 page 109 on microphone 8.
\item Character Eulalie on microphone 9.
\item Character Man \#1 page 108 and Man \#1 page 109 on microphone 13.
\item Character Mayor Shinn on microphone 17.
\item Character Woman page 109 on microphone 18.
\end{enumerate}
Note that all microphones are now off.
\subsection* {109: Harold enters at end of chase}
\subsubsection* {Switch On}
\begin{enumerate}
\item Character Harold on microphone 1 using DCA 0 named Harold.
\item Character Marian on microphone 2 using DCA 1 named Marian.
\end{enumerate}
\subsection* {110: ``Please , hurry, please ---''}
\subsubsection* {Switch On}
\begin{enumerate}
\item Character Marcellus on microphone 15 using DCA 6 named Marcellus.
\end{enumerate}
\subsection* {110: ``Let's try down by the crick!''}
\subsubsection* {Switch On}
\begin{enumerate}
\item Character Winthrop on microphone 14 using DCA 4 named Winthrop.
\end{enumerate}
\subsubsection* {Switch Off}
\begin{enumerate}
\item Character Marcellus on microphone 15.
\end{enumerate}
\subsubsection* {Still On}
Note that the following microphones are still on:
\begin{enumerate}
\item Character Harold on microphone 1 using DCA 0 named Harold.
\item Character Marian on microphone 2 using DCA 1 named Marian.
\item Character Winthrop on microphone 14 using DCA 4 named Winthrop.
\end{enumerate}
\subsection* {113: ``till there was you.''}
\subsubsection* {Switch On}
\begin{enumerate}
\item Character Marcellus on microphone 15 using DCA 6 named Marcellus.
\end{enumerate}
\subsection* {113: ``that way --- that way!''}
\subsubsection* {Switch Off}
\begin{enumerate}
\item Character Harold on microphone 1.
\item Character Marian on microphone 2.
\item Character Winthrop on microphone 14.
\item Character Marcellus on microphone 15.
\end{enumerate}
Note that all microphones are now off.
\subsection* {113: Scene seven opens}
\subsubsection* {Switch On}
\begin{enumerate}
\item Character Mayor Shinn on microphone 17 using DCA 3 named Mayor Shinn.
\end{enumerate}
\subsection* {114: ``Four Score ---''}
\subsubsection* {Switch On}
\begin{enumerate}
\item Character Man \#1 page 115 on microphone 12 in group Men Act 2 using DCA 6 named Men Act 2.
\end{enumerate}
\subsection* {114: ``get our money back''}
\subsubsection* {Switch On}
\begin{enumerate}
\item Character Man \#2 page 114 on microphone 23 in group Men Act 2 using DCA 6 named Men Act 2.
\end{enumerate}
\subsubsection* {Switch Off}
\begin{enumerate}
\item Character Man \#1 page 115 on microphone 12.
\end{enumerate}
\subsubsection* {Still On}
Note that the following microphones are still on:
\begin{enumerate}
\item Character Mayor Shinn on microphone 17 using DCA 3 named Mayor Shinn.
\item Character Man \#2 page 114 on microphone 23 using DCA 6 named Men Act 2.
\end{enumerate}
\subsection* {114: ``for uniforms, just tonight!''}
\subsubsection* {Switch On}
\begin{enumerate}
\item Character Woman \#1 page 114 on microphone 11 in group Women Act 2 using DCA 5 named Women Act 2.
\end{enumerate}
\subsubsection* {Switch Off}
\begin{enumerate}
\item Character Man \#2 page 114 on microphone 23.
\end{enumerate}
\subsubsection* {Still On}
Note that the following microphones are still on:
\begin{enumerate}
\item Character Woman \#1 page 114 on microphone 11 using DCA 5 named Women Act 2.
\item Character Mayor Shinn on microphone 17 using DCA 3 named Mayor Shinn.
\end{enumerate}
\subsection* {114: ``seen them uniforms yet!'}
\subsubsection* {Switch Off}
\begin{enumerate}
\item Character Woman \#1 page 114 on microphone 11.
\end{enumerate}
\subsubsection* {Still On}
Note that character Mayor Shinn on microphone 17 is still on, using DCA 3 named Mayor Shinn.\subsection* {114: ``He's slippery.~ I told you ---''}
\subsubsection* {Switch On}
\begin{enumerate}
\item Character Woman \#2 page 114 on microphone 22 in group Women Act 2 using DCA 5 named Women Act 2.
\end{enumerate}
\subsection* {114: ``since just after supper!''}
\subsubsection* {Switch On}
\begin{enumerate}
\item Character Man \#3 page 114 on microphone 10 in group Men Act 2 using DCA 6 named Men Act 2.
\end{enumerate}
\subsubsection* {Switch Off}
\begin{enumerate}
\item Character Woman \#2 page 114 on microphone 22.
\end{enumerate}
\subsubsection* {Still On}
Note that the following microphones are still on:
\begin{enumerate}
\item Character Man \#3 page 114 on microphone 10 using DCA 6 named Men Act 2.
\item Character Mayor Shinn on microphone 17 using DCA 3 named Mayor Shinn.
\end{enumerate}
\subsection* {114: ``He's a kidnapper!''}
\subsubsection* {Switch On}
\begin{enumerate}
\item Character Woman \#3 page 114 on microphone 24 in group Women Act 2 using DCA 5 named Women Act 2.
\end{enumerate}
\subsubsection* {Switch Off}
\begin{enumerate}
\item Character Man \#3 page 114 on microphone 10.
\end{enumerate}
\subsubsection* {Still On}
Note that the following microphones are still on:
\begin{enumerate}
\item Character Mayor Shinn on microphone 17 using DCA 3 named Mayor Shinn.
\item Character Woman \#3 page 114 on microphone 24 using DCA 5 named Women Act 2.
\end{enumerate}
\subsection* {114: ``Fine situation here!''}
\subsubsection* {Switch Off}
\begin{enumerate}
\item Character Woman \#3 page 114 on microphone 24.
\end{enumerate}
\subsubsection* {Still On}
Note that character Mayor Shinn on microphone 17 is still on, using DCA 3 named Mayor Shinn.\subsection* {114: ``Four Score ---''}
\subsubsection* {Switch On}
\begin{enumerate}
\item Character Harold on microphone 1 using DCA 0 named Harold.
\item Character Marian on microphone 2 using DCA 1 named Marian.
\end{enumerate}
\subsection* {115: Mayor Shinn shakes hands with Tommy}
\subsubsection* {Switch On}
\begin{enumerate}
\item Character Man \#1 page 115 on microphone 12 in group Men Act 2 using DCA 6 named Men Act 2.
\item Character Maud on microphone 20 using DCA 2 named Maud.
\item Character Alma on microphone 21 using DCA 5 named Alma.
\item Character Man \#2 page 115 on microphone 23 in group Men Act 2 using DCA 6 named Men Act 2.
\end{enumerate}
\subsection* {116: Harold embraces Marian}
\subsubsection* {Switch Off}
\begin{enumerate}
\item Character Harold on microphone 1.
\item Character Marian on microphone 2.
\item Character Man \#1 page 115 on microphone 12.
\item Character Mayor Shinn on microphone 17.
\item Character Maud on microphone 20.
\item Character Alma on microphone 21.
\item Character Man \#2 page 115 on microphone 23.
\end{enumerate}
Note that all microphones are now off.


\subsection {Shoe Tree Labels}

When not being worn by an actor, the microphones are stored in a shoe tree,
which is 4 shoes wide by 6 shoes tall.  
Each slot has a space at the top 4.75 inches wide
by 1.5 inches high.  We provide here a label for each slot showing which
actors use that microphone.

\embedfile [desc={input file}] {Music_Man.txt.microphone_labels.tex}
{\Large
\parbox[t]{4.75in}{1\hfil\break Stuart Harmon}\hfil\break\vskip 0.25in
\parbox[t]{4.75in}{2\hfil\break Ashley Hughes}\hfil\break\vskip 0.25in
\parbox[t]{4.75in}{3\hfil\break Roger Hurd}\hfil\break\vskip 0.25in
\parbox[t]{4.75in}{4\hfil\break Russell Arrowsmith}\hfil\break\vskip 0.25in
\parbox[t]{4.75in}{5\hfil\break Nicholas Hammes}\hfil\break\vskip 0.25in
\parbox[t]{4.75in}{6\hfil\break Steve Hammes\hfil\break Nancy Bennison}\hfil\break\vskip 0.25in
\parbox[t]{4.75in}{7\hfil\break Tom Partridge\hfil\break Robyn Holley}\hfil\break\vskip 0.25in
\parbox[t]{4.75in}{8\hfil\break Ken Cox\hfil\break Lucy Stover}\hfil\break\vskip 0.25in
\parbox[t]{4.75in}{9\hfil\break Meg Petersen\hfil\break Joe Cinque}\hfil\break\vskip 0.25in
\parbox[t]{4.75in}{10\hfil\break Olivia Vordenberg\hfil\break Gordon Goyette}\hfil\break\vskip 0.25in
\parbox[t]{4.75in}{11\hfil\break Mike McKnight\hfil\break Jane Curran}\hfil\break\vskip 0.25in
\parbox[t]{4.75in}{12\hfil\break Bob Piotrowski\hfil\break Janine Leffler}\hfil\break\vskip 0.25in
\parbox[t]{4.75in}{13\hfil\break Patrick McKnight\hfil\break Rich Sparks}\hfil\break\vskip 0.25in
\parbox[t]{4.75in}{14\hfil\break Ben Wesenberg\hfil\break Dave Ostrowski}\hfil\break\vskip 0.25in
\parbox[t]{4.75in}{15\hfil\break Mark Sousa\hfil\break Kevin Linkroum\hfil\break Sophie Linkroum}\hfil\break\vskip 0.25in
\parbox[t]{4.75in}{16\hfil\break Michelle Emmond\hfil\break Pat Sawicki}\hfil\break\vskip 0.25in
\parbox[t]{4.75in}{17\hfil\break Vick Bennison}\hfil\break\vskip 0.25in
\parbox[t]{4.75in}{18\hfil\break Christa Vordenberg\hfil\break Kim Whitehead\hfil\break Megan Ostrowski\hfil\break Theresa Novek}\hfil\break\vskip 0.25in
\parbox[t]{4.75in}{19\hfil\break Mark Merrifield}\hfil\break\vskip 0.25in
\parbox[t]{4.75in}{20\hfil\break Paula Troie\hfil\break Matt O'Dowd}\hfil\break\vskip 0.25in
\parbox[t]{4.75in}{21\hfil\break Jenn Fichera\hfil\break Janet Scagnelli}\hfil\break\vskip 0.25in
\parbox[t]{4.75in}{22\hfil\break Jenn Stover\hfil\break JoAnn Zall\hfil\break Donna Fallon}\hfil\break\vskip 0.25in
\parbox[t]{4.75in}{23\hfil\break Steve Taylor\hfil\break Kevin Brannelly}\hfil\break\vskip 0.25in
\parbox[t]{4.75in}{24\hfil\break Karen Hammes\hfil\break Susan Abis\hfil\break Becky Tripp}\hfil\break\vskip 0.25in

}

\subsection {Front of House operator}

The Front of House operator controls the house mixer, 
which contains a volume slider
and mute button for each microphone.  

\subsubsection {Microphone Labels}

Associated with each slider is space to write a name.  
These spaces are 1 inch wide by 0.75 inch tall.  
Here is a label for each space:

\embedfile [desc={input file}] {Music_Man.txt.microphone_channels.tex}
{\Large
\vskip 0.25in \vbox to0.75in {\parbox[t]{1in}{Harold\vfil}\vfil}
\vskip 0.25in \vbox to0.75in {\parbox[t]{1in}{Marian\vfil}\vfil}
\vskip 0.25in \vbox to0.75in {\parbox[t]{1in}{Jacey\vfil}\vfil}
\vskip 0.25in \vbox to0.75in {\parbox[t]{1in}{Ewart\vfil}\vfil}
\vskip 0.25in \vbox to0.75in {\parbox[t]{1in}{Oliver\vfil}\vfil}
\vskip 0.25in \vbox to0.75in {\parbox[t]{1in}{Olin\vfil}\vfil}
\vskip 0.25in \vbox to0.75in {\parbox[t]{1in}{Charlie Cowell\vfil}\vfil}
\vskip 0.25in \vbox to0.75in {\parbox[t]{1in}{Conductor\vfil}\vfil}
\vskip 0.25in \vbox to0.75in {\parbox[t]{1in}{Eulalie\vfil}\vfil}
\vskip 0.25in \vbox to0.75in {\parbox[t]{1in}{Zaneeta\vfil}\vfil}
\vskip 0.25in \vbox to0.75in {\parbox[t]{1in}{Salesman \#5\vfil}\vfil}
\vskip 0.25in \vbox to0.75in {\parbox[t]{1in}{Salesman \#4\vfil}\vfil}
\vskip 0.25in \vbox to0.75in {\parbox[t]{1in}{Tommy\vfil}\vfil}
\vskip 0.25in \vbox to0.75in {\parbox[t]{1in}{Winthrop\vfil}\vfil}
\vskip 0.25in \vbox to0.75in {\parbox[t]{1in}{Marcellus\vfil}\vfil}
\vskip 0.25in \vbox to0.75in {\parbox[t]{1in}{Mrs.~Paroo\vfil}\vfil}
\vskip 0.25in \vbox to0.75in {\parbox[t]{1in}{Mayor Shinn\vfil}\vfil}
\vskip 0.25in \vbox to0.75in {\parbox[t]{1in}{Amaryllis\vfil}\vfil}
\vskip 0.25in \vbox to0.75in {\parbox[t]{1in}{Man page 11\vfil}\vfil}
\vskip 0.25in \vbox to0.75in {\parbox[t]{1in}{Maud\vfil}\vfil}
\vskip 0.25in \vbox to0.75in {\parbox[t]{1in}{Alma\vfil}\vfil}
\vskip 0.25in \vbox to0.75in {\parbox[t]{1in}{Ethel\vfil}\vfil}
\vskip 0.25in \vbox to0.75in {\parbox[t]{1in}{Constable Locke\vfil}\vfil}
\vskip 0.25in \vbox to0.75in {\parbox[t]{1in}{Mrs.~Squires\vfil}\vfil}

}

\subsubsection {Microphone Switching Instructions by Page}

These instructions are intended to be pasted into the script
for users of manual sound boards.

\embedfile [desc={input file}] {Music_Man.txt.microphone_switching.tex}
\vskip 0.25in \vbox {\parbox[t]{2.5in}{{\textbf{Top of Show}} On: 7, 8, 9 (Train passengers), 10 (Train passengers), 11 (Train passengers), 12 (Train passengers), 13 (Train passengers). }}\vfil
\vskip 0.25in \vbox {\parbox[t]{2.5in}{{\textbf{2: ``Boart!  All aboardt!''}} Off: 8. (Still on: 7, 9, 10, 11, 12, 13.)}}\vfil
\vskip 0.25in \vbox {\parbox[t]{2.5in}{{\textbf{6: ``Hill?''}} On: 14 (Train passengers), 15 (Train passengers). }}\vfil
\vskip 0.25in \vbox {\parbox[t]{2.5in}{{\textbf{6: ``No!''}} Off: 14, 15. (Still on: 7, 9, 10, 11, 12, 13.)}}\vfil
\vskip 0.25in \vbox {\parbox[t]{2.5in}{{\textbf{8: ``but he doesn't know the territory!''}} On: 8. }}\vfil
\vskip 0.25in \vbox {\parbox[t]{2.5in}{{\textbf{8: ``Seegarettes illegal in this state.~ Board!''}} Off: 8. (Still on: 7, 9, 10, 11, 12, 13.)}}\vfil
\vskip 0.25in \vbox {\parbox[t]{2.5in}{{\textbf{10: ``sell them neck-bowed Hawkeyes out here.''}} On: 1, 8. }}\vfil
\vskip 0.25in \vbox {\parbox[t]{2.5in}{{\textbf{10: ``I don't believe I caught your name.''}} Off: 1, 7, 8, 9, 10, 11, 12, 13. (All off.)}}\vfil
\vskip 0.25in \vbox {\parbox[t]{2.5in}{{\textbf{10: immediately following}} On: 20 (Boys). }}\vfil
\vskip 0.25in \vbox {\parbox[t]{2.5in}{{\textbf{11: ``you really ought to give Iowa a try,''}} On: 6, 17, 19 (Men), 21, 22. }}\vfil
\vskip 0.25in \vbox {\parbox[t]{2.5in}{{\textbf{11: ``She wouldn't a' come anyway.''}} Off: 6, 17, 19, 20, 21, 22. (All off.)}}\vfil
\vskip 0.25in \vbox {\parbox[t]{2.5in}{{\textbf{12: ``If your crop should happen to die.''}} On: 23 (Men), 24 (Women). }}\vfil
\vskip 0.25in \vbox {\parbox[t]{2.5in}{{\textbf{12: ``Even though we may not ever mention it again.''}} Off: 23, 24. (All off.)}}\vfil
\vskip 0.25in \vbox {\parbox[t]{2.5in}{{\textbf{12: ``ought to give Iowa a try.''}} On: 1, 3 (Jacey). }}\vfil
\vskip 0.25in \vbox {\parbox[t]{2.5in}{{\textbf{13: ``Who is late as usual.''}} On: 25. Off: 3. (Still on: 1, 25.)}}\vfil
\vskip 0.25in \vbox {\parbox[t]{2.5in}{{\textbf{13: ``Music teacher.''}} On: 4 (Ewart). Off: 25. (Still on: 1, 4.)}}\vfil
\vskip 0.25in \vbox {\parbox[t]{2.5in}{{\textbf{15: ``a pool table in your community.''}} Off: 4. (Still on: 1.)}}\vfil
\vskip 0.25in \vbox {\parbox[t]{2.5in}{{\textbf{22: ``The young ones moral after school''}} On: 2. }}\vfil
\vskip 0.25in \vbox {\parbox[t]{2.5in}{{\textbf{23: ``Good!''}} Off: 1, 2. (All off.)}}\vfil
\vskip 0.25in \vbox {\parbox[t]{2.5in}{{\textbf{23: immediately following}} On: 2, 16, 18. }}\vfil
\vskip 0.25in \vbox {\parbox[t]{2.5in}{{\textbf{27: ``Yes, dear.''}} On: 26. }}\vfil
\vskip 0.25in \vbox {\parbox[t]{2.5in}{{\textbf{28: ``He's crying.''}} Off: 16, 26. (Still on: 2, 18.)}}\vfil
\vskip 0.25in \vbox {\parbox[t]{2.5in}{{\textbf{31: ``Goodnight.''}} Off: 2, 18. (All off.)}}\vfil
\vskip 0.25in \vbox {\parbox[t]{2.5in}{{\textbf{31: Scene five opens}} On: 27. }}\vfil
\vskip 0.25in \vbox {\parbox[t]{2.5in}{{\textbf{31: song ends}} On: 17. Off: 27. (Still on: 17.)}}\vfil
\vskip 0.25in \vbox {\parbox[t]{2.5in}{{\textbf{32: ``will take a leading part.''}} On: 27. }}\vfil
\vskip 0.25in \vbox {\parbox[t]{2.5in}{{\textbf{33: ``I'm shot, George; who shot me?''}} On: 21, 23, 32. Off: 27. (Still on: 17, 21, 23, 32.)}}\vfil
\vskip 0.25in \vbox {\parbox[t]{2.5in}{{\textbf{33: ``Four store and seven ---''}} On: 3 (Quartet), 4 (Quartet), 5 (Quartet), 6 (Quartet). }}\vfil
\vskip 0.25in \vbox {\parbox[t]{2.5in}{{\textbf{34: ``Four score ---''}} On: 1, 11 (Men), 12 (Men), 28. Off: 3, 4, 5, 6, 21, 23, 32. (Still on: 1, 11, 12, 17, 28.)}}\vfil
\vskip 0.25in \vbox {\parbox[t]{2.5in}{{\textbf{35: ``we get this pool table matter settled!''}} Off: 11, 12, 17, 28. (Still on: 1.)}}\vfil
\vskip 0.25in \vbox {\parbox[t]{2.5in}{{\textbf{38: Shinn corners the school board}} On: 17, 23, 30. }}\vfil
\vskip 0.25in \vbox {\parbox[t]{2.5in}{{\textbf{39: ``Ten o'clock sharp.''}} Off: 17. (Still on: 1, 23, 30.)}}\vfil
\vskip 0.25in \vbox {\parbox[t]{2.5in}{{\textbf{40: ``What's your name?''}} On: 31. }}\vfil
\vskip 0.25in \vbox {\parbox[t]{2.5in}{{\textbf{41: ``Ye Gods.''}} Off: 30, 31. (Still on: 1, 23.)}}\vfil
\vskip 0.25in \vbox {\parbox[t]{2.5in}{{\textbf{41: ``She's the Mayor's oldest girl.''}} On: 3 (Quartet), 4 (Quartet), 5 (Quartet), 6 (Quartet), 27. Off: 23. (Still on: 1, 3, 4, 5, 6, 27.)}}\vfil
\vskip 0.25in \vbox {\parbox[t]{2.5in}{{\textbf{44: ``mystify...''}} Off: 1. (Still on: 3, 4, 5, 6, 27.)}}\vfil
\vskip 0.25in \vbox {\parbox[t]{2.5in}{{\textbf{44: ``Oh, we're in love.''}} Off: 3, 4, 5, 6, 27. (All off.)}}\vfil
\vskip 0.25in \vbox {\parbox[t]{2.5in}{{\textbf{44: Scene six opens}} On: 1, 2. }}\vfil
\vskip 0.25in \vbox {\parbox[t]{2.5in}{{\textbf{45: ``interesting information about you.''}} On: 25. Off: 2. (Still on: 1, 25.)}}\vfil
\vskip 0.25in \vbox {\parbox[t]{2.5in}{{\textbf{48: ``The sadder but wiser girl for me.''}} On: 21, 22, 27, 28, 29. Off: 25. (Still on: 1, 21, 22, 27, 28, 29.)}}\vfil
\vskip 0.25in \vbox {\parbox[t]{2.5in}{{\textbf{52: ``cheep, cheep, cheep, cheep!''}} On: 3 (Quartet), 4 (Quartet), 5 (Quartet), 6 (Quartet). }}\vfil
\vskip 0.25in \vbox {\parbox[t]{2.5in}{{\textbf{53: ``pick a little, talk a little, cheep!''}} Off: 1, 3, 4, 5, 6, 21, 22, 27, 28, 29. (All off.)}}\vfil
\vskip 0.25in \vbox {\parbox[t]{2.5in}{{\textbf{53: Scene seven opens}} On: 1, 2. }}\vfil
\vskip 0.25in \vbox {\parbox[t]{2.5in}{{\textbf{54: ``No!''}} Off: 2. (Still on: 1.)}}\vfil
\vskip 0.25in \vbox {\parbox[t]{2.5in}{{\textbf{56: Marian slaps Tommy}} Off: 1. (All off.)}}\vfil
\vskip 0.25in \vbox {\parbox[t]{2.5in}{{\textbf{57: Scene eight opens}} On: 1, 30. }}\vfil
\vskip 0.25in \vbox {\parbox[t]{2.5in}{{\textbf{57: ``G'bye Professor.''}} Off: 30. (Still on: 1.)}}\vfil
\vskip 0.25in \vbox {\parbox[t]{2.5in}{{\textbf{57: doorbell rings}} On: 17. }}\vfil
\vskip 0.25in \vbox {\parbox[t]{2.5in}{{\textbf{59: ``a Quaker on his day off.''}} Off: 1, 17. (All off.)}}\vfil
\vskip 0.25in \vbox {\parbox[t]{2.5in}{{\textbf{59: Scene ten opens}} On: 1, 16, 26. }}\vfil
\vskip 0.25in \vbox {\parbox[t]{2.5in}{{\textbf{60: ``What do you think of that?''}} Off: 26. (Still on: 1, 16.)}}\vfil
\vskip 0.25in \vbox {\parbox[t]{2.5in}{{\textbf{61: ``Hodado, Miss Paroo.''}} On: 2. }}\vfil
\vskip 0.25in \vbox {\parbox[t]{2.5in}{{\textbf{62: ``Well Miss Paroo, I hardly ---''}} Off: 2. (Still on: 1, 16.)}}\vfil
\vskip 0.25in \vbox {\parbox[t]{2.5in}{{\textbf{62: ``good day to ya, Widda Paroo.''}} On: 2. Off: 1. (Still on: 2, 16.)}}\vfil
\vskip 0.25in \vbox {\parbox[t]{2.5in}{{\textbf{63: ``I know you're there.''}} On: 26. }}\vfil
\vskip 0.25in \vbox {\parbox[t]{2.5in}{{\textbf{63: ``I've written it all down.''}} Off: 26. (Still on: 2, 16.)}}\vfil
\vskip 0.25in \vbox {\parbox[t]{2.5in}{{\textbf{63: ``He does?''}} Off: 16. (Still on: 2.)}}\vfil
\vskip 0.25in \vbox {\parbox[t]{2.5in}{{\textbf{64: ```Til I die.''}} Off: 2. (All off.)}}\vfil
\vskip 0.25in \vbox {\parbox[t]{2.5in}{{\textbf{65: Scene eleven opens}} On: 30, 31. }}\vfil
\vskip 0.25in \vbox {\parbox[t]{2.5in}{{\textbf{66: ``Tommy!  It's Papa!''}} On: 2, 17, 27. Off: 30, 31. (Still on: 2, 17, 27.)}}\vfil
\vskip 0.25in \vbox {\parbox[t]{2.5in}{{\textbf{67: ``It's on page...''}} On: 32. }}\vfil
\vskip 0.25in \vbox {\parbox[t]{2.5in}{{\textbf{67: ``The band instruments!''}} Off: 2, 17, 27, 32. (All off.)}}\vfil
\vskip 0.25in \vbox {\parbox[t]{2.5in}{{\textbf{67: ``I wish, I wish I knew what it could be.''}} On: 9 (WFW soloists), 14 (WFW soloists), 15 (WFW soloists), 19 (WFW soloists). }}\vfil
\vskip 0.25in \vbox {\parbox[t]{2.5in}{{\textbf{68: ``and a cross-cut saw.''}} Off: 9, 14, 15, 19. (All off.)}}\vfil
\vskip 0.25in \vbox {\parbox[t]{2.5in}{{\textbf{68: ``It is a prepaid surprise or C.O.D.?''}} On: 18 (WFW soloists), 20 (WFW soloists), 24 (WFW soloists), 28 (WFW soloists). }}\vfil
\vskip 0.25in \vbox {\parbox[t]{2.5in}{{\textbf{68: ``or a double boiler,''}} Off: 20, 24, 28. (Still on: 18.)}}\vfil
\vskip 0.25in \vbox {\parbox[t]{2.5in}{{\textbf{68: ``just for me.''}} Off: 18. (All off.)}}\vfil
\vskip 0.25in \vbox {\parbox[t]{2.5in}{{\textbf{69: ``I wish I knew what he was comin' for.''}} On: 3 (Quartet), 4 (Quartet), 5 (Quartet), 6 (Quartet), 11 (WFW soloists), 13 (WFW soloists), 23 (WFW soloists). }}\vfil
\vskip 0.25in \vbox {\parbox[t]{2.5in}{{\textbf{69: ``for the courthouse square.''}} On: 26. Off: 3, 4, 5, 6, 11, 13, 23. (Still on: 26.)}}\vfil
\vskip 0.25in \vbox {\parbox[t]{2.5in}{{\textbf{69: ``Ray-yy!''}} On: 10. }}\vfil
\vskip 0.25in \vbox {\parbox[t]{2.5in}{{\textbf{69: ``Whoa!''}} Off: 10. (Still on: 26.)}}\vfil
\vskip 0.25in \vbox {\parbox[t]{2.5in}{{\textbf{70: ``It'th the band insthrumenth!''}} On: 1, 17. }}\vfil
\vskip 0.25in \vbox {\parbox[t]{2.5in}{{\textbf{70: ``O thithter!''}} On: 27. Off: 26. (Still on: 1, 17, 27.)}}\vfil
\vskip 0.25in \vbox {\parbox[t]{2.5in}{{\textbf{71: ``About that book ---''}} Off: 17, 27. (Still on: 1.)}}\vfil
\vskip 0.25in \vbox {\parbox[t]{2.5in}{{\textbf{71: ``Tuesday nights at the High School.''}} Off: 1. (All off.)}}\vfil
\vskip 0.25in \vbox {\parbox[t]{2.5in}{{\textbf{73: Act two scene one opens}} On: 27. }}\vfil
\vskip 0.25in \vbox {\parbox[t]{2.5in}{{\textbf{73: ``All right, Mr.~Dunlop.''}} On: 3 (Quartet), 4 (Quartet), 5 (Quartet), 6 (Quartet). }}\vfil
\vskip 0.25in \vbox {\parbox[t]{2.5in}{{\textbf{74: ``Oh, yes, ``it's you''''}} On: 25, 30. Off: 3, 4, 5, 6. (Still on: 25, 27, 30.)}}\vfil
\vskip 0.25in \vbox {\parbox[t]{2.5in}{{\textbf{74: ``we are entitled to five more minutes.''}} Off: 27. (Still on: 25, 30.)}}\vfil
\vskip 0.25in \vbox {\parbox[t]{2.5in}{{\textbf{75: ``Shipoopi!''}} Off: 30. (Still on: 25.)}}\vfil
\vskip 0.25in \vbox {\parbox[t]{2.5in}{{\textbf{77: ``but you can win her yet.''}} Off: 25. (All off.)}}\vfil
\vskip 0.25in \vbox {\parbox[t]{2.5in}{{\textbf{77: dancing concludes}} On: 23 (Boys). }}\vfil
\vskip 0.25in \vbox {\parbox[t]{2.5in}{{\textbf{77: ``show us some new steps!''}} Off: 23. (All off.)}}\vfil
\vskip 0.25in \vbox {\parbox[t]{2.5in}{{\textbf{77: ``Shipoopi!''}} On: 17, 27, 30, 31. }}\vfil
\vskip 0.25in \vbox {\parbox[t]{2.5in}{{\textbf{78: ``Get out, you wild kid.''}} Off: 30. (Still on: 17, 27, 31.)}}\vfil
\vskip 0.25in \vbox {\parbox[t]{2.5in}{{\textbf{79: ``My dance ---''}} On: 1, 2. Off: 27. (Still on: 1, 2, 17, 31.)}}\vfil
\vskip 0.25in \vbox {\parbox[t]{2.5in}{{\textbf{80: ``He's slipprier'n a Mississippi sturgeon!''}} On: 5 (Quartet). }}\vfil
\vskip 0.25in \vbox {\parbox[t]{2.5in}{{\textbf{80: ``a button hook in the well-water.''}} Off: 5, 17, 31. (Still on: 1, 2.)}}\vfil
\vskip 0.25in \vbox {\parbox[t]{2.5in}{{\textbf{80: ``Why any night this week ---''}} On: 21, 22, 28, 29. Off: 1. (Still on: 2, 21, 22, 28, 29.)}}\vfil
\vskip 0.25in \vbox {\parbox[t]{2.5in}{{\textbf{83: ``Rabelais!''}} On: 27. }}\vfil
\vskip 0.25in \vbox {\parbox[t]{2.5in}{{\textbf{84: (whispered) ``Cheep!''}} Off: 2, 21, 22, 27, 28, 29. (All off.)}}\vfil
\vskip 0.25in \vbox {\parbox[t]{2.5in}{{\textbf{84: Scene two opens}} On: 1, 3 (Quartet), 4 (Quartet), 5 (Quartet), 6 (Quartet). }}\vfil
\vskip 0.25in \vbox {\parbox[t]{2.5in}{{\textbf{85: ``I'm home again, rose.''}} Off: 1. (Still on: 3, 4, 5, 6.)}}\vfil
\vskip 0.25in \vbox {\parbox[t]{2.5in}{{\textbf{85: ``Oh, Lida Rose.''}} On: 2. }}\vfil
\vskip 0.25in \vbox {\parbox[t]{2.5in}{{\textbf{87: ``Oh, Lida Rose.''}} Off: 3, 4, 5, 6. (Still on: 2.)}}\vfil
\vskip 0.25in \vbox {\parbox[t]{2.5in}{{\textbf{87: Scene three opens}} On: 16. }}\vfil
\vskip 0.25in \vbox {\parbox[t]{2.5in}{{\textbf{88: ``there's nothing wrong with a ladylike hint.''}} On: 26. }}\vfil
\vskip 0.25in \vbox {\parbox[t]{2.5in}{{\textbf{90: ``La de da, La de da.''}} Off: 26. (Still on: 2, 16.)}}\vfil
\vskip 0.25in \vbox {\parbox[t]{2.5in}{{\textbf{90: ``There's time later.''}} On: 7. Off: 16. (Still on: 2, 7.)}}\vfil
\vskip 0.25in \vbox {\parbox[t]{2.5in}{{\textbf{93: ``Try me.''}} On: 3 (Quartet), 4 (Quartet), 5 (Quartet), 6 (Quartet). }}\vfil
\vskip 0.25in \vbox {\parbox[t]{2.5in}{{\textbf{94: ``Neither one of you's heard the last of me, girly-girl!''}} On: 1, 16. Off: 7. (Still on: 1, 2, 3, 4, 5, 6, 16.)}}\vfil
\vskip 0.25in \vbox {\parbox[t]{2.5in}{{\textbf{94: ``Oh, Lida Rose.''}} Off: 3, 4, 5, 6. (Still on: 1, 2, 16.)}}\vfil
\vskip 0.25in \vbox {\parbox[t]{2.5in}{{\textbf{95: ``Well, I'll put some on.''}} Off: 16. (Still on: 1, 2.)}}\vfil
\vskip 0.25in \vbox {\parbox[t]{2.5in}{{\textbf{100: ``Fifteen minutes.''}} On: 16. Off: 1. (Still on: 2, 16.)}}\vfil
\vskip 0.25in \vbox {\parbox[t]{2.5in}{{\textbf{100: ``I been using the Think System on you from the Parlor!''}} Off: 2, 16. (All off.)}}\vfil
\vskip 0.25in \vbox {\parbox[t]{2.5in}{{\textbf{101: Scene four opens}} On: 1, 25. }}\vfil
\vskip 0.25in \vbox {\parbox[t]{2.5in}{{\textbf{102: ``I'll meet you at the Hotel in plenty of time.''}} On: 2. Off: 25. (Still on: 1, 2.)}}\vfil
\vskip 0.25in \vbox {\parbox[t]{2.5in}{{\textbf{104: ``there's a lot of things you don't know about me ---''}} On: 25. }}\vfil
\vskip 0.25in \vbox {\parbox[t]{2.5in}{{\textbf{104: ``Go get the rig.''}} Off: 25. (Still on: 1, 2.)}}\vfil
\vskip 0.25in \vbox {\parbox[t]{2.5in}{{\textbf{107: ``My someone, goodnight.''}} On: 7, 25. Off: 2. (Still on: 1, 7, 25.)}}\vfil
\vskip 0.25in \vbox {\parbox[t]{2.5in}{{\textbf{107: ``you never even knew the territory.''}} Off: 7. (Still on: 1, 25.)}}\vfil
\vskip 0.25in \vbox {\parbox[t]{2.5in}{{\textbf{107: ``Come on! Hurry up!''}} Off: 1, 25. (All off.)}}\vfil
\vskip 0.25in \vbox {\parbox[t]{2.5in}{{\textbf{108: Scene six opens}} On: 7, 8 (Men), 13 (Men), 17, 18 (Women), 27. }}\vfil
\vskip 0.25in \vbox {\parbox[t]{2.5in}{{\textbf{109: ``After him!''}} Off: 7, 8, 13, 17, 18, 27. (All off.)}}\vfil
\vskip 0.25in \vbox {\parbox[t]{2.5in}{{\textbf{109: Harold enters at end of chase}} On: 1, 2. }}\vfil
\vskip 0.25in \vbox {\parbox[t]{2.5in}{{\textbf{110: ``Please , hurry, please ---''}} On: 25. }}\vfil
\vskip 0.25in \vbox {\parbox[t]{2.5in}{{\textbf{110: ``Let's try down my the crick!''}} On: 26. Off: 25. (Still on: 1, 2, 26.)}}\vfil
\vskip 0.25in \vbox {\parbox[t]{2.5in}{{\textbf{113: ``till there was you.''}} On: 25. }}\vfil
\vskip 0.25in \vbox {\parbox[t]{2.5in}{{\textbf{113: ``that way --- that way!''}} Off: 1, 2, 25, 26. (All off.)}}\vfil
\vskip 0.25in \vbox {\parbox[t]{2.5in}{{\textbf{113: Scene seven opens}} On: 17. }}\vfil
\vskip 0.25in \vbox {\parbox[t]{2.5in}{{\textbf{114: ``Four Score ---''}} On: 12 (Men). }}\vfil
\vskip 0.25in \vbox {\parbox[t]{2.5in}{{\textbf{114: ``get our money back''}} On: 9 (Men). Off: 12. (Still on: 9, 17.)}}\vfil
\vskip 0.25in \vbox {\parbox[t]{2.5in}{{\textbf{114: ``for uniforms, just tonight!''}} On: 14 (Women). Off: 9. (Still on: 14, 17.)}}\vfil
\vskip 0.25in \vbox {\parbox[t]{2.5in}{{\textbf{114: ``seen them uniforms yet!'}} Off: 14. (Still on: 17.)}}\vfil
\vskip 0.25in \vbox {\parbox[t]{2.5in}{{\textbf{114: ``He's slippery.~ I told you ---''}} On: 24 (Women). }}\vfil
\vskip 0.25in \vbox {\parbox[t]{2.5in}{{\textbf{114: ``since just after supper!''}} On: 10 (Men). Off: 24. (Still on: 10, 17.)}}\vfil
\vskip 0.25in \vbox {\parbox[t]{2.5in}{{\textbf{114: ``He's a kidnapper!''}} On: 11 (Women). Off: 10. (Still on: 11, 17.)}}\vfil
\vskip 0.25in \vbox {\parbox[t]{2.5in}{{\textbf{114: ``Fine situation here!''}} Off: 11. (Still on: 17.)}}\vfil
\vskip 0.25in \vbox {\parbox[t]{2.5in}{{\textbf{114: ``Four Score ---''}} On: 1, 2. }}\vfil
\vskip 0.25in \vbox {\parbox[t]{2.5in}{{\textbf{115: Mayor Shinn shakes hands with Tommy}} On: 9 (Men), 12 (Men), 21, 28. }}\vfil
\vskip 0.25in \vbox {\parbox[t]{2.5in}{{\textbf{116: Harold embraces Marian}} Off: 1, 2, 9, 12, 17, 21, 28. (All off.)}}\vfil


\end{document}
