\documentclass[letterpaper]{article}
\usepackage[utf8]{inputenc}
\usepackage[T1]{fontenc}
\usepackage[english]{babel}
\usepackage{color}
\usepackage[top=0.5in,bottom=0.75in,left=0.5in,right=0.5in,includehead,head=12pt,headsep=0.2in,includefoot,foot=12pt,footskip=0.25in]{geometry}
\usepackage{multicol}
\usepackage{array}
\usepackage{longtable}
% Pages styles
\makeatletter
\newcommand\ps@Standard{
  \renewcommand\@oddhead{Microphones for The Music Man\hfill April 26, 2011}
  \renewcommand\@evenhead{\@oddhead}
  \renewcommand\@oddfoot{\hfil Page {\thepage} \hfil}
  \renewcommand\@evenfoot{\@oddfoot}
  \renewcommand\thepage{\arabic{page}}
}
\makeatother
\pagestyle{Standard}
\setlength\tabcolsep{1mm}
\renewcommand\arraystretch{1.3}
\title{Microphones for The Music Man}
\author{John Sauter}
\date{2011-04-26}
\begin{document}
\maketitle
\tableofcontents
\newpage

\section {Microphones}
In general, whenever an actor speaks, or sings by himself, he has a microphone.
I have made a few compromises to reduce the number of microphones to 22,
which I thought was the maximum number the theater could handle.
It turned out that the maximum was 20, so I added a second mixer, which
increased the number to 24.  I kept the compromises in because it
provided more time for microphone moves and meant that more actors
never had to share microphones.

\subsection {Assignments}
Microphones are assigned to actors as follows:

\begin{center}
\begin{longtable}{|l|m{7in}|}
\hline Mic. & Assignment \endhead \hline
\input music_man.txt.assignments.tex
\end{longtable}
\end{center}

\subsection {Microphone Moves, by Page}

\begin{center}
\begin{longtable}{|l|m{7in}|}
\hline page & moves \endhead \hline
\input music_man.txt.microphone_moves.tex
\end{longtable}
\end{center}

\subsection {Microphone Handling Instructions for Each Actor}

These instructions are formatted to be sliced into a strip of paper
for each of the actors to carry in order to remember his instructions.
In addition, the complete list can be posted near the shoe tree that
holds the microphones.

\vskip 0.25in

{\setlength{\parindent}{0in}
\input music_man.txt.actors.tex
}

\subsection {List of Microphone Transactions for the Stage Manager}

\subsubsection {Initial Microphone Assignments}

These actors are the first to enter with their microphones:

\begin{center}
\begin{longtable}{|l|l|}
\hline Mic. & Actor \endhead \hline
\input music_man.txt.stage_manager_before_show.tex
\end{longtable}
\end{center}

\subsubsection {Microphone Exchanges}

As the show progresses, some actors will have to share microphones. The
following list shows when microphones become free as actors leave the
stage, and when they go back into use on a different actor.

\begin{center}
\begin{longtable}{|l|l|l|l|}
\hline page & Mic. & Direction & Actor \endhead \hline
\input music_man.txt.stage_manager_during_show.tex
\end{longtable}
\end{center}

\subsubsection {End of Show}

At the end of the show, retrieve the microphones from these actors

\begin{center}
\begin{longtable}{|l|l|}
\hline Mic. & Actor \endhead \hline
\input music_man.txt.stage_manager_after_show.tex
\end{longtable}
\end{center}

\subsubsection {Intermission}

In case we start a rehearsal at the intermission, here are the microphone
assignments at the end of intermission.  This list can also be useful as a 
cross-check during a production.

\begin{center}
\begin{longtable}{|l|l|}
\hline Mic. & Actor \endhead \hline
\input music_man.txt.stage_manager_at_intermission.tex
\end{longtable}
\end{center}

\subsubsection {Shoe Tree Labels}

When not being worn by an actor, the 24 microphones are stored in a shoe tree,
which is 4 shoes wide by 6 shoes tall.  
Each slot has a space at the top 4.75 inches wide
by 1.5 inches high.  We provide here a label for each slot showing which
actors use that microphone, and when.

{\Large
\input music_man.txt.microphone_labels.tex
}

\subsection {Front of House operator}

The Front of House operator controls the house mixer, 
which contains a volume slider
and mute button for each microphone.  

\subsubsection {Microphone Labels}

Associated with each slider is space to write a name.  
These spaces are 1 inch wide by 0.75 inch tall.  
Here is a label for each space:

{\Large
\input music_man.txt.microphone_channels.tex
}

\subsubsection {Microphone Switching Instructions by Page}

These instructions are intended to be pasted into the script.

\input music_man.txt.microphone_switching.tex

\end{document}
